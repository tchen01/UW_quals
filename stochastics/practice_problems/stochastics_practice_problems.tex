\documentclass[12pt]{article}
\usepackage[T1]{fontenc}

% Document Details
\newcommand{\CLASS}{Stochastics}
\newcommand{\assigmentnum}{Methods and Problems}


\usepackage[margin = 1in, top = 1.25in, bottom = 1.in]{geometry}
\usepackage{titling}
\setlength{\droptitle}{-6em}   % This is your set screw
\date{}
\renewcommand{\maketitle}{
	\clearpage
	\begingroup  
	\centering
	\LARGE \sffamily\textbf{\CLASS} \Large \assigmentnum\\[.8em]
	\large Tyler Chen\\[1em]
	\endgroup
	\thispagestyle{empty}
}
 % Title Styling
\usepackage{tocloft}
\renewcommand{\cfttoctitlefont}{\Large\sffamily\bfseries}
\renewcommand{\cftsecfont}{\normalfont\sffamily\bfseries}
\renewcommand{\cftsubsecfont}{\normalfont\sffamily}
\renewcommand{\cftsubsubsecfont}{\normalfont\sffamily}

\makeatletter
\let\oldl@section\l@section
\def\l@section#1#2{\oldl@section{#1}{\sffamily\bfseries#2}}

\let\oldl@subsection\l@subsection
\def\l@subsection#1#2{\oldl@subsection{#1}{\sffamily#2}}

\let\oldl@subsubsection\l@subsubsection
\def\l@subsubsection#1#2{\oldl@subsubsection{#1}{\sffamily#2}}
 % ToC Styling


\usepackage{enumitem}

% Figures
\usepackage{subcaption}

% TikZ and Graphics
\usepackage{tikz, pgfplots}
\pgfplotsset{compat=1.12}
\usetikzlibrary{patterns}
\usepgfplotslibrary{fillbetween}

\usepackage{pdfpages}
\usepackage{adjustbox}

\usepackage{lscape}
\usepackage{titling}
\usepackage[]{hyperref}


% Header Styling
\usepackage{fancyhdr}
\pagestyle{fancy}
\lhead{\sffamily \CLASS}
\rhead{\sffamily Chen \textbf{\thepage}}
\cfoot{}

% Paragraph Styling
\setlength{\columnsep}{1cm}
\setlength{\parindent}{0pt}
\setlength{\parskip}{5pt}
\renewcommand{\baselinestretch}{1}

% TOC Styling
\usepackage{tocloft}
\iffalse
\renewcommand{\cftsecleader}{\cftdotfill{\cftdotsep}}

\renewcommand\cftchapafterpnum{\vskip6pt}
\renewcommand\cftsecafterpnum{\vskip10pt}
\renewcommand\cftsubsecafterpnum{\vskip6pt}

% Adjust sectional unit title fonts in ToC
\renewcommand{\cftchapfont}{\sffamily}
\renewcommand{\cftsecfont}{\bfseries\sffamily}
\renewcommand{\cftsecnumwidth}{2em}
\renewcommand{\cftsubsecfont}{\sffamily}
\renewcommand{\cfttoctitlefont}{\hfill\bfseries\sffamily\MakeUppercase}
\renewcommand{\cftaftertoctitle}{\hfill}

\renewcommand{\cftchappagefont}{\sffamily}
\renewcommand{\cftsecpagefont}{\bfseries\sffamily}
\renewcommand{\cftsubsecpagefont}{\sffamily}
\fi
 % General Styling

% Section Styling
\usepackage{sectsty}

\allsectionsfont{\normalfont\sffamily\bfseries}
%\sectionfont{\sffamily}
%\subsectionfont{\sffamily}
%\subsubsectionfont{\sffamily}


\renewcommand{\abstractname}{\sffamily\large ABSTRACT}
 % Section Styling
% Code Display Setup
\usepackage{listings,lstautogobble}
\usepackage{lipsum}
\usepackage{courier}
\usepackage{catchfilebetweentags}

\lstset{
	basicstyle=\small\ttfamily,
	breaklines=true, 
	frame = single,
	rangeprefix=,
	rangesuffix=,
	includerangemarker=false,
	autogobble = true
}


\usepackage{algorithmicx}
\usepackage{algpseudocode}

\newcommand{\To}{\textbf{to}~}
\newcommand{\DownTo}{\textbf{downto}~}
\renewcommand{\algorithmicdo}{\hspace{-.2em}\textbf{:}}
 % Code Display Setup
% AMS MATH Styling
\usepackage{amsmath, amssymb}
\newcommand{\qed}{\hfill\(\square\)}

%\newtheorem*{lemma}{Lemma} 
%\newtheorem*{theorem}{Theorem}
%\newtheorem*{definition}{Definition}
%\newtheorem*{prop}{Proposition}
%\renewenvironment{proof}{{\bfseries Proof.}}{}


% mathcal
\newcommand{\cA}{\ensuremath{\mathcal{A}}}
\newcommand{\cB}{\ensuremath{\mathcal{B}}}
\newcommand{\cC}{\ensuremath{\mathcal{C}}}
\newcommand{\cD}{\ensuremath{\mathcal{D}}}
\newcommand{\cE}{\ensuremath{\mathcal{E}}}
\newcommand{\cF}{\ensuremath{\mathcal{F}}}
\newcommand{\cG}{\ensuremath{\mathcal{G}}}
\newcommand{\cH}{\ensuremath{\mathcal{H}}}
\newcommand{\cI}{\ensuremath{\mathcal{I}}}
\newcommand{\cJ}{\ensuremath{\mathcal{J}}}
\newcommand{\cK}{\ensuremath{\mathcal{K}}}
\newcommand{\cL}{\ensuremath{\mathcal{L}}}
\newcommand{\cM}{\ensuremath{\mathcal{M}}}
\newcommand{\cN}{\ensuremath{\mathcal{N}}}
\newcommand{\cO}{\ensuremath{\mathcal{O}}}
\newcommand{\cP}{\ensuremath{\mathcal{P}}}
\newcommand{\cQ}{\ensuremath{\mathcal{Q}}}
\newcommand{\cR}{\ensuremath{\mathcal{R}}}
\newcommand{\cS}{\ensuremath{\mathcal{S}}}
\newcommand{\cT}{\ensuremath{\mathcal{T}}}
\newcommand{\cU}{\ensuremath{\mathcal{U}}}
\newcommand{\cV}{\ensuremath{\mathcal{V}}}
\newcommand{\cW}{\ensuremath{\mathcal{W}}}
\newcommand{\cX}{\ensuremath{\mathcal{X}}}
\newcommand{\cY}{\ensuremath{\mathcal{Y}}}
\newcommand{\cZ}{\ensuremath{\mathcal{Z}}}

% mathbb
\usepackage{bbm}
\newcommand{\bOne}{\ensuremath{\mathbbm{1}}}

\newcommand{\bA}{\ensuremath{\mathbb{A}}}
\newcommand{\bB}{\ensuremath{\mathbb{B}}}
\newcommand{\bC}{\ensuremath{\mathbb{C}}}
\newcommand{\bD}{\ensuremath{\mathbb{D}}}
\newcommand{\bE}{\ensuremath{\mathbb{E}}}
\newcommand{\bF}{\ensuremath{\mathbb{F}}}
\newcommand{\bG}{\ensuremath{\mathbb{G}}}
\newcommand{\bH}{\ensuremath{\mathbb{H}}}
\newcommand{\bI}{\ensuremath{\mathbb{I}}}
\newcommand{\bJ}{\ensuremath{\mathbb{J}}}
\newcommand{\bK}{\ensuremath{\mathbb{K}}}
\newcommand{\bL}{\ensuremath{\mathbb{L}}}
\newcommand{\bM}{\ensuremath{\mathbb{M}}}
\newcommand{\bN}{\ensuremath{\mathbb{N}}}
\newcommand{\bO}{\ensuremath{\mathbb{O}}}
\newcommand{\bP}{\ensuremath{\mathbb{P}}}
\newcommand{\bQ}{\ensuremath{\mathbb{Q}}}
\newcommand{\bR}{\ensuremath{\mathbb{R}}}
\newcommand{\bS}{\ensuremath{\mathbb{S}}}
\newcommand{\bT}{\ensuremath{\mathbb{T}}}
\newcommand{\bU}{\ensuremath{\mathbb{U}}}
\newcommand{\bV}{\ensuremath{\mathbb{V}}}
\newcommand{\bW}{\ensuremath{\mathbb{W}}}
\newcommand{\bX}{\ensuremath{\mathbb{X}}}
\newcommand{\bY}{\ensuremath{\mathbb{Y}}}
\newcommand{\bZ}{\ensuremath{\mathbb{Z}}}

% alternative mathbb
\newcommand{\NN}{\ensuremath{\mathbb{N}}}
\newcommand{\RR}{\ensuremath{\mathbb{R}}}
\newcommand{\CC}{\ensuremath{\mathbb{C}}}
\newcommand{\ZZ}{\ensuremath{\mathbb{Z}}}
\newcommand{\EE}{\ensuremath{\mathbb{E}}}
\newcommand{\PP}{\ensuremath{\mathbb{P}}}
\newcommand{\VV}{\ensuremath{\mathbb{V}}}
\newcommand{\cov}{\ensuremath{\text{Co}\VV}}
% Math Commands

\newcommand{\st}{~\big|~}
\newcommand{\stt}{\text{ st. }}
\newcommand{\ift}{\text{ if }}
\newcommand{\thent}{\text{ then }}
\newcommand{\owt}{\text{ otherwise }}

\newcommand{\norm}[1]{\left\lVert#1\right\rVert}
\newcommand{\ip}[1]{\ensuremath{\left\langle #1 \right\rangle}}
\newcommand{\pp}[3][]{\frac{\partial^{#1}#2}{\partial #3^{#1}}}
\newcommand{\dd}[3][]{\frac{\d^{#1}#2}{\d #3^{#1}}}
\renewcommand{\d}{\ensuremath{\mathrm{d}}}

\newcommand{\indep}{\rotatebox[origin=c]{90}{$\models$}}




 % Math shortcuts
% Problem
\usepackage{mdframed}
\usepackage{floatrow}

\newenvironment{problem}[1][]
{\pagebreak
\begin{mdframed}[
  frametitle={\sffamily #1},
  linewidth=1,
  topline=false,
  bottomline=false,
  rightline=false,
  rightmargin=.5cm
]}
{\end{mdframed}}

\newenvironment{solution}[1][]
{\begin{mdframed}[
  frametitle={\sffamily #1},
  linecolor=gray!70,
  linewidth=1,
  topline=false,
  bottomline=false,
  rightline=false,
  rightmargin=.5cm
]}
{\end{mdframed}}



 % Math shortcuts
\usepackage{floatrow}
\usepackage{mdframed}

\newenvironment{algorithm}[1][\@nil]
{\def\tmp{#1}%
\begin{mdframed}[
  frametitle={Algorithm. \ifx\tmp\@nnil  \else \normalfont (#1) \fi},
  linecolor=green!70,
  linewidth=1,
  topline=false,
  bottomline=false,
  rightline=false,
  rightmargin=.5cm
]}
{\end{mdframed}}

\newenvironment{method}[1][\@nil]
{
\def\tmp{#1}%
\begin{mdframed}[
  frametitle={Method. \ifx\tmp\@nnil  \else \normalfont (#1) \fi},
  linecolor=violet!70,
  linewidth=1,
  topline=false,
  bottomline=false,
  rightline=false,
  rightmargin=.5cm
]}
{\end{mdframed}}

\newenvironment{definition}[1][\@nil]
{\def\tmp{#1}%
\begin{mdframed}[
  frametitle={Definition. \ifx\tmp\@nnil  \else \normalfont (#1) \fi},
  linecolor=blue!70,
  linewidth=1,
  topline=false,
  bottomline=false,
  rightline=false,
  rightmargin=.5cm
]}
{\end{mdframed}}

\newenvironment{theorem}[1][\@nil]
{\def\tmp{#1}%
\begin{mdframed}[
  frametitle={Theorem. \ifx\tmp\@nnil  \else \normalfont (#1) \fi},
  linecolor=red!70,
  linewidth=1,
  topline=false,
  bottomline=false,
  rightline=false,
  rightmargin=.5cm
]}
{\end{mdframed}}

\newenvironment{lemma}[1][\@nil]
{\def\tmp{#1}%
\begin{mdframed}[
  frametitle={Lemma. \ifx\tmp\@nnil  \else \normalfont (#1) \fi},
  linecolor=red!70,
  linewidth=1,
  topline=false,
  bottomline=false,
  rightline=false,
  rightmargin=.5cm
]}
{\end{mdframed}}

\newenvironment{proof}[1][\@nil]
{\def\tmp{#1}%
\begin{mdframed}[
  frametitle={Proof. \ifx\tmp\@nnil  \else \normalfont (#1) \fi},
  linecolor=red!30,
  linewidth=1,
  topline=false,
  bottomline=false,
  rightline=false,
  rightmargin=.5cm
]}
{\end{mdframed}}



 % Math shortcuts


\hypersetup{
    colorlinks=true,       % false: boxed links; true: colored links
    linkcolor=violet,          % color of internal links (change box color with linkbordercolor)
    citecolor=green,        % color of links to bibliography
    filecolor=magenta,      % color of file links
    urlcolor=cyan           % color of external links
}

\setlength{\headheight}{15pt}
\newcommand{\note}[1]{\textcolor{red}{\textit{Note:} #1}}

% overwrite old problem class to be able to add to ToC
\let\savedprob=\problem%
\def\problem[#1]{\pagebreak\phantomsection\addcontentsline{toc}{subsection}{#1}\savedprob[#1]\label{#1}}


\begin{document}
\maketitle

\pagebreak
\tableofcontents

%%%%%%%%%%%%%%%%%%%%%
%    Useful Info    %
%%%%%%%%%%%%%%%%%%%%%
\pagebreak
\section{Table of Random Varibles and Distributions}

\textit{Probability Mass Function} (for discrete random variables):
\begin{align*}
    p(k) = \PP(X = k)
\end{align*}

\textit{Probability Density Function} (for continuous random variables):
\begin{align*}
    p(x)\d x = \PP(X \in [x,x+\d x))
\end{align*}

\textit{Probability Generating Function}:
\begin{align*}
    G(z) = \EE[z^X] = p(0) + p(1)z + p(2)z^2 + p(3)z^3 + \cdots
\end{align*}

\textit{Characteristic Function}:
\begin{align*}
    \phi(t) = \EE[e^{itX}] 
\end{align*}



\subsection{Bernoulli}
Models if a heads is flipped for a biased coin.
\begin{center}
\def\arraystretch{1.5}
\begin{tabular}{|r|l|} \hline
    Parameters & \( p\in[0,1] \) \\ \hline
    Support & \( \{0,1\} \) \\ \hline
    PMF & \( \begin{cases} 1-p & k=0 \\ p & k=1 \end{cases} \) \\ \hline 
    Mean & \( p \) \\ \hline
    Variance & \( p(1-p) \) \\ \hline
    PGF & \( (1-p)+pz \) \\ \hline
    CF & \( (1-p)+pe^{it} \)\\ \hline
\end{tabular}
\end{center}


\subsection{Binomial}
Models the number of heads when flipping a biased coin \( n \) times.
\begin{center}
\def\arraystretch{1.5}
\begin{tabular}{|r|l|} \hline
    Parameters & \( p\in[0,1],n\in \NN_{\geq 0} \) \\ \hline
    Support & \( \{0,1,\ldots, n\} \) \\ \hline
    PMF & \( \binom{n}{k} p^k(1-p)^{n-k} \) \\ \hline 
    Mean & \( np \) \\ \hline
    Variance & \( np(1-p) \) \\ \hline
    PGF & \( [(1-p)+pz]^n \) \\ \hline
    CF & \( [(1-p) + pe^{it}]^n \)\\ \hline
\end{tabular}
\end{center}


\subsection{Geometric}
Models the number of flips of a biased coin required to flip a heads.
\begin{center}
\def\arraystretch{1.5}
\begin{tabular}{|r|l|} \hline
    Parameters & \( p\in[0,1] \) \\ \hline
    Support & \( \{1,\ldots, n\} \) \\ \hline
    PMF & \( p(1-p)^{k-1} \) \\ \hline 
    CDF & \( 1-(1-p)^k \) \\ \hline
    Mean & \( 1/p \) \\ \hline
    Variance & \( (1-p)/p^2 \) \\ \hline
    PGF & \( ps/(1-(1-p)s) \) \\ \hline
    CF & \( pe^{it}/(1-(1-p)e^{it}) \)\\ \hline
\end{tabular}
\end{center}

\subsection{Poisson}
Expresses the probability of a given number of events occurring in a fixed interval of time or space if these events occur with a known constant rate and independently of the time since the last event.
\begin{center}
\def\arraystretch{1.5}
\begin{tabular}{|r|l|} \hline
    Parameters & \( \lambda > 0 \) \\ \hline
    Support & \( \{0,1,2,\ldots\} \) \\ \hline
    PMF & \( \lambda^k e^{-\lambda}/k! \) \\ \hline 
    CDF & \( e^{-\lambda} \sum_{j=0}^{k} \lambda^j/j! \) \\ \hline
    Mean & \( \lambda \) \\ \hline
    Variance & \( \lambda \) \\ \hline
    PGF & \( \exp(\lambda(z-1)) \) \\ \hline
    CF & \( \exp(\lambda(e^{it}-1)) \)\\ \hline
\end{tabular}
\end{center}


\pagebreak
\section{Table of Random Processes}

give probability of jump in time dt





%%%%%%%%%%%%%%%%%%%%%
%     CHAPTER 3     %
%%%%%%%%%%%%%%%%%%%%%
\pagebreak
\section{Generating and Characteristic functions}

how to get density from gen function



%%%%%%%%%%%%%%%%%%%%%
%     CHAPTER 4     %
%%%%%%%%%%%%%%%%%%%%%
\pagebreak
\section{Discrete Time Markov Chains}
\subsection{Transition Matrix}
\textit{Sample Problems}: 
\begin{itemize}[nolistsep]
    \item \hyperref[Exercise 4.1]{Exercise 4.1}: Write down transition matrices for processes based on rolling a dice
    \item \hyperref[Exercise 4.2]{Exercise 4.2}: Write down transition matrices for \( Y_n = X_{2n} \)
    \item \hyperref[Exercise 4.7]{Exercise 4.7}: Give example of transition matrix with multiple stationary distributions
\end{itemize}

\subsection{Classification of States}


\textit{Sample Problems}: 
\begin{itemize}[nolistsep]
    \item \hyperref[Exercise 4.3]{Exercise 4.3}: Show if all states communicate with an absorbing state they must all be transient
\end{itemize}

\subsection{Mean Recurence Time}

\textit{Sample Problems}: 
\begin{itemize}[nolistsep]
    \item \hyperref[Exercise 4.4]{Exercise 4.4}: Find expected visits to a state given some properties
    \item \hyperref[Exercise 4.5]{Exercise 4.5}: Find mean-recurrence times using invariant distribution
\end{itemize}

\subsection{Reversibility}
\textit{Sample Problems}: 
\begin{itemize}[nolistsep]
    \item \hyperref[Exercise 4.8]{Exercise 4.8}: Show process is reversible in equilibrium
\end{itemize}


\subsection{Stationary/Invariant distribution}

\note{TALK ABOUT VARIOUS METHODS FOR FINDING THIS}

\textit{Sample Problems}: 
\begin{itemize}[nolistsep]
    \item \hyperref[Exercise 4.5]{Exercise 4.5}: Find invariant distribution
    \item \hyperref[Exercise 4.6]{Exercise 4.6}: Find invariant distribution of mistakes in editions of a book by computing limit of generating function
    \item \hyperref[Exercise 4.7]{Exercise 4.7}: Give example of transition matrix with multiple stationary distributions
\end{itemize}

\subsection{Generating Functions}
\textit{Sample Problems}: 
\begin{itemize}[nolistsep]
    \item \hyperref[Exercise 4.6]{Exercise 4.6}: Find invariant distribution of mistakes in editions of a book by computing limit of generating function
\end{itemize}





\pagebreak
%%%%%%%%%%%%%%%%%%%%%
%     CHAPTER 5     %
%%%%%%%%%%%%%%%%%%%%%
\section{Continuous Time Markov Chains}

\subsection{Transition Matrix}


\subsection{Stationary/Invariant distribution}

\textit{Sample Problems}: 
\begin{itemize}[nolistsep]
    \item \hyperref[Exercise 5.1]{Exercise 5.1}: Find invariant distribution and conditions for existence
    \item \hyperref[Exercise 5.2]{Exercise 5.2}: Show two processes have the same stationary distribution 
    \item \hyperref[Exercise 5.3]{Exercise 5.3}: Indirectly find stationary distribution by solving KFE, finding generating function for the chain, and computing the distribution of \( X_t \) as \( t\to\infty \)
\end{itemize}

\subsection{Generator}
\textit{Sample Problems}: 
\begin{itemize}[nolistsep]
    \item \hyperref[Exercise 5.1]{Exercise 5.1}: Write down generator
    \item \hyperref[Exercise 5.3]{Exercise 5.3}: Given generator solve KFE
    \item \hyperref[Exercise 5.4]{Exercise 5.4}: Write down generator and solve KFE/KBE
\end{itemize}

\subsection{Generating Functions}
\textit{Sample Problems}: 
\begin{itemize}[nolistsep]
    \item \hyperref[Exercise 5.3]{Exercise 5.3}: Use KBE to find PDE for generating function of \( X \)
    \item \hyperref[Exercise 5.4]{Exercise 5.4}: Use KBE to find PDE for generating function of \( X \)
    \item \hyperref[Exercise 5.5]{Exercise 5.5}: Compute generating function of Poisson process with random intensity. Use generating function to compute mean and variance.
\end{itemize}



\subsection{KFE AND KBE}
\textit{Sample Problems}: 
\begin{itemize}[nolistsep]
    \item \hyperref[Exercise 5.3]{Exercise 5.3}: Given generator solve KFE
    \item \hyperref[Exercise 5.4]{Exercise 5.4}: Write down KFE and KBE and solve
\end{itemize}



\subsection{Birth Death Processes}

General description of birth death processes

\subsubsection{General Form for infinite queue}
\textit{Description}:
\begin{itemize}[nolistsep]
    \item Process either jumps up one or down one or stay the same
    \item Expected wait time in state \( i \) is exponentially distributed \( \tau \sim \cE( \lambda_i + \mu_i) \)
    \item When the process does jump, the probability of an up jump is \( \lambda_i / (\lambda_i+\mu_i) \), and the probability of a down jump is \( \mu_i / (\lambda_i+\mu_i) \).
    \item if \( \lambda_0 > 0 \) the chain is irreducible.
\end{itemize}


\textit{State space}: \( S = \{1,2,3\ldots\}  \).

\textit{Generator}:
\begin{align*}
    G = \left[\begin{array}{cccccc}
        -\lambda_0 & \lambda_0 \\
        \mu_1 & -(\mu_1+\lambda_1) & \lambda_1 \\
        & \mu_2 & -(\mu_2+\lambda_2) & \lambda_2 \\
        && \mu_3 & -(\mu_3+\lambda_3) & \lambda_3 \\
        &&&  & \ddots &  
    \end{array}\right]
\end{align*}


\textit{Invariant distribution}:
\begin{align*}
    \pi(k) = \frac{\lambda_0 \lambda_1 \cdots \lambda_{k-1}}{\mu_1 \mu_2 \cdots \mu_k} \pi(0), 
    && \pi(0) = \left( 1+ \sum_{k=1}^{\infty}   \frac{\lambda_0 \lambda_1 \cdots \lambda_{k-1}}{\mu_1 \mu_2 \cdots \mu_k}  \right)^{-1}
\end{align*}


\textit{Sample Problems}: Example 5.2.9



\subsubsection{M/M/1 queue}
\textit{Description}:
\begin{itemize}[nolistsep]
\item Models infinite queue. 
\item Arrivals occur at a rate \( \lambda \) according to a Poisson process. 
\item Service times have exponential distribution with rate parameter \( \mu \), where \( 1/\mu \) is the mean service time.
\item A single server serves customers one at a time from front of queue, first come first serve
\end{itemize}


\textit{State space}: \( S = \{1,2,3\ldots\}  \).

\textit{Generator}:
\begin{align*}
    G = \left[\begin{array}{ccccc}
        -\lambda & \lambda \\
        \mu & -(\mu+\lambda) & \lambda \\
        & \mu & -(\mu+\lambda) & \lambda \\
        &&  & \ddots &  
    \end{array}\right]
\end{align*}


\textit{Invariant distribution}:
\begin{align*}
    \pi(k) = (1-\lambda/\mu)(\lambda/\mu)^k
\end{align*}

\textit{Expected Response Time}:
For customers who arrive and find the queue as a stationary process, the response time (sum of waiting and services times) has density function,
\begin{align*}
    f(t) = \begin{cases}
        (\mu-\lambda)e^{-(\mu-\lambda)t}, & t > 0 \\ 0 & \text{ow.}
    \end{cases} 
\end{align*}
This has mean,
\begin{align*}
    \int_0^\infty tf(t)\d t = \frac{1}{\mu - \lambda}
\end{align*}


\textit{Sample Problems}: \hyperref[Exercise 5.1]{Exercise 5.1}


\subsubsection{M/M/\(\infty\)}
\textit{Description}:
\begin{itemize}[nolistsep]
\item Arrivals occur at a rate \( \lambda \) according to a Poisson process. 
\item Service times have exponential distribution with rate parameter \( \mu \), where \( 1/\mu \) is the mean service time.
\item There are always enough servers that every arriving job is serviced immediately.
\end{itemize}


\textit{State space}: \( S = \{1,2,3,\ldots\} \).

\textit{Generator}:
\begin{align*}
    G = \left[\begin{array}{cccccc}
        -\lambda & \lambda \\
        \mu & -(\mu+\lambda) & \lambda \\
        & 2\mu & -(2\mu+\lambda) & \lambda \\
        & & 3\mu & -(3\mu+\lambda) & \lambda \\
        && & & \ddots 
    \end{array}\right]
\end{align*}

\textit{Invariant Distribution}:
\begin{align*}
    \pi(k) = \frac{(\lambda/\mu)^ke^{-\lambda/\mu}}{k!}
\end{align*}


\textit{Sample Problems}: \hyperref[Exercise 5.3]{Exercise 5.3}, Final Problem ??, Practice Exam \#? Problem 1


\subsubsection{M/M/1/K queue}

\textit{State space}: \( S = \{1,2,\ldots, n\} \).

\textit{Generator}:
\begin{align*}
    G = \left[\begin{array}{cccccc}
        -\lambda & \lambda \\
        \mu & -(\mu+\lambda) & \lambda \\
        & \mu & -(\mu+\lambda) & \lambda \\
        \\
        && \ddots & \ddots & \ddots \\
        \\
        &&& \mu & -(\mu+\lambda) & \lambda \\
        &&&& \mu & -\mu
    \end{array}\right]
\end{align*}


%%%%%%%%%%%%%%%%%%%%%
%     CHAPTER 7     %
%%%%%%%%%%%%%%%%%%%%%
\pagebreak
\section{Brownian Motion}
\note{add examples from class notes}

\subsection{Martingale}
\textit{Sample Problems}: 
\begin{itemize}[nolistsep]
    \item \hyperref[Exercise 7.1]{Exercise 7.1}: Show a process is a Martingale using definition
    \item \hyperref[Exercise 7.4]{Exercise 7.4}: Show a process is a Martingale using definition
\end{itemize}

\subsection{Characteristic Functions}
\textit{Sample Problems}: 
\begin{itemize}[nolistsep]
    \item \hyperref[Exercise 7.2]{Exercise 7.2}: Compute characteristic function of \( W(N(t)) \), where \( N\sim \operatorname{Pois}(\lambda) \)
\end{itemize}

7.3: n-th variation time


\subsection{Laplace Transform}
\textit{Sample Problems}: 
\begin{itemize}[nolistsep]
    \item \note{Example ???} from book
    \item \hyperref[Exercise 7.4]{Exercise 7.4}: Compute Laplace transform of first hitting time.
\end{itemize}


%%%%%%%%%%%%%%%%%%%%%
%     CHAPTER 8     %
%%%%%%%%%%%%%%%%%%%%%
\pagebreak
\section{Stochastic Calculus}

\note{ITO FORMULA AND STUFF}
\subsection{It\^o's Formula}
\textit{One Dimension}:
\begin{align*}
    \d f(X_t) = f'(X_t)\d X_t + \frac{1}{2}f''(X_t)\d[X,X]_t
\end{align*}

\textit{Two Dimensions}:
\begin{align*}
    \d f(t,X_t) = f_t(t,X_t)\d t + f_x(t,X_t)\d X_t + \frac{1}{2}f_{xx}(t,X_t) \d[X,X]_t
\end{align*}

\textit{Two Dimensions}:
\begin{align*}
    \d f(X_t,Y_t) &= f_x(X_t,Y_t)\d X_t + f_y(X_t,Y_t)\d Y_t 
    \\&\hspace{3em} + \frac{1}{2} \Big( f_{xx}(X_t,Y_t)\d[X,X]_t + f_{xy}(X_t,Y_t)\d[X,Y]_t 
    \\&\hspace{6em}+ f_{yx}(X_t,Y_t)\d[Y,X]_t + f_{yy}(X_t,Y_t)\d[Y,Y]_t \Big)
\end{align*}


%%%%%%%%%%%%%%%%%%%%%
%     CHAPTER 9     %
%%%%%%%%%%%%%%%%%%%%%
\pagebreak
\section{SDEs and PDEs}

\note{ADD ASSOCIATED PDEs}

\subsection{Geometric Brownian Motion}


\subsection{Ornstein--Uhlenbeck (OU) process}
\textit{SDE}:
\begin{align*}
    \d X_t = \kappa(\theta-X_t)\d t + \d W_t
\end{align*}

\textit{Solution}:
\begin{align*}
    X_t = \theta + e^{-\kappa t}(X_0-\theta) + \int_0^t e^{-\kappa(t-s)}\d W_s
\end{align*}



%%%%%%%%%%%%%%%%%%%%%
%     CHAPTER 10    %
%%%%%%%%%%%%%%%%%%%%%
\pagebreak
\section{Jump Diffusions}



\pagebreak
\section{Practice Qualification Exams}
\begin{problem}[Practice Exam 1, Problem 1]
    Let \( X = (X_n)_{n\in\NN_0} \) be a discrete time Markov chain with \( X_n \) representig the amount of water in a reservoir at noon on day \( n \). Assume \( X_0 \in \NN_0 \). Let \( Y = (Y_n)_{n\in\NN_0} \) be a sequence of iid random variables with \( Y_n \) representing the aount of water that flows into the reservoir during the \( n \)-th day. The state space of \( Y \) is \( \{0,1,2,\ldots \} \). The resevoir has a maximum capacity of \( K\in\NN \). When the resevoir is filled to level \( K \), all excssive inflows are lost.
    \begin{enumerate}[nolistsep,label=(\alph*)]
        \item Write the one-step transition matrix \( P \) of \( X \) in terms of the probability generating function \( G_Y \) of \( Y \).
        \item Find an expression for the stationary distribution \( \pi \) of \( X \) in terms of the probability generating function \( G_Y \) of \( Y \).
    \end{enumerate}
\end{problem}

\begin{solution}[Solution]
\begin{enumerate}[label=(\alph*)]
    \item 
        We assume all the water comes in the afternoon. That is, \( X_{n+1} = X_n + Y_n \).

        Suppose on day \( n \) the resevoir is not full. That is, \( X_n = k < K \). If it is not filled completely by the incoming water, then some amount of water \( j < K-k \) was added. In this case \( X_{n+1} = k+j \) with probability,
        \begin{align*}
            \PP(Y_n = j) = f_Y(j) = 
            \left[\frac{1}{j!}\dd[j]{G_Y(s)}{s} \right]_{s=0}
        \end{align*}
        
        Otherwise, \( X_{n+1} = K \) with probability,
        \begin{align*}
            1-\sum_{j < K-k} f_Y(j) = 1 - \sum_{j<K-k} \left[\frac{1}{j!} \dd[j]{G_Y(s)}{s} \right]_{s=0}
        \end{align*}
        
        Suppose \( X_n = K \). Then since no water leaves the resevoir, \( X_{n+1} = K \) with probability one.

        We can write this as,
        \begin{align*}
            X_{n+1} = \begin{cases}
                0 & j < i \\
                \left[\frac{1}{j!}\dd[j]{G_Y(s)}{s} \right]_{s=0} & j < K - X_n \\ \\
                1 - \sum_{j<K-X_n} \left[\frac{1}{j!}\dd[j]{G_Y(s)}{s} \right]_{s=0} & \text{otherwise}
            \end{cases}
        \end{align*}
       
        Conditioning on \( X_n = i \) we then have,
        \begin{align*}
            P_{i,j} = \PP(X_{n+1} = j | X_n = i) = 
             \begin{cases}
                0 & j < i \\
                \left[\frac{1}{j!}\dd[j]{G_Y(s)}{s} \right]_{s=0} & j < K - i \\ \\
                1 - \sum_{j<i} \left[\frac{1}{j!}\dd[j]{G_Y(s)}{s} \right]_{s=0} & \text{otherwise}
            \end{cases}
        \end{align*}


    \item
        Note that \( \pi = [0,0,\ldots,0,1] \) is a stationary distribution.

        \note{argue the distributoin is unique?}


        \note{alternative approach??}
        Clearly \( X_n \to K \) as \( n\to\infty \).

        \note{in what sense?}
        
\end{enumerate}
\end{solution}

\begin{problem}[Practice Exam 1, Problem 2]
    Let \( (X,Y) = (X_t,Y_t)_{t\geq 0} \) satisfy the following SDE,
    \begin{align*}
        \d X_t = \d W_t^1, && \d Y_t = \d W_t^2, && (X_0,Y_0) = (x,y)
    \end{align*}
    where \( W = (W_t^1,W_t^2)_{t\geq 0} \) is a two-dimensinoal Brownian motion with independent components. Define a process \( (R,\Phi) = (R_t,\Phi_t)_{t\geq 0} \) as follows,
    \begin{align*}
        \Phi_t = \arctan(Y_t/X_t), && R_t^2 = X_t^2 + Y_t^2
    \end{align*}
    \begin{enumerate}[nolistsep,label=(\alph*)]
        \item Derive the SDEs satisfied by \( (R,\Phi) \).
        \item Define,
            {\small
            \begin{align*}
                u(r,\phi) = \EE \left[ e^{-\lambda \tau} f(R_\tau) | R_0 = r,\Phi_0 = \phi \right], &&
                \tau = \inf\{t\geq 0:\Phi_t\notin(0,\pi/2)\}, &&
                \phi \in (0,\pi/2)
            \end{align*}
            }
            Derive a PDE satisfied by \( u \).
        \item Desribe with pseudo-code how you would find \( u(r,\phi) \) using Monte Carlo simulation.
    \end{enumerate}    
\end{problem}

\begin{solution}[Solution]
\begin{enumerate}[label=(\alph*)]
    \item Define \( f(x,y) = \arctan(y/x) \) and \( g(x,y) = \sqrt{x^2+y^2} \). Now note that,
        \begin{align*}
            \Phi_t = f(X_t,Y_t), && R_t = g(X_t,Y_t)
        \end{align*}

        Appying It\^o's formula we find,
        \begin{align*}
            \d \Phi_t &=  f_x(X_t,Y_t)\d X_t + f_y(X_t,Y_t)\d Y_t 
            \\&\hspace{3em}+ \frac{1}{2}\big(f_{xx}(X_t,Y_t)\d[X,X]_t + f_{xy}(X_t,Y_t)\d [X,Y]_t 
            \\&\hspace{6em}+ f_{yx}(X_t,Y_t)\d[Y,X]_t + f_{yy}(X_t,Y_t)\d[Y,Y]_t\big) 
        \end{align*}
        
        Using our Heuristics we have,
        \begin{align*}
            \d[X,X]_t = \d[Y,Y]_t = \d t, && \d[X,Y]_T = \d[Y,X]_t = 0
        \end{align*}
        
        We compute,
        \begin{align*}
            f_x(x,y) &= -\frac{y}{x^2+y^2} = - \frac{\sin(\arctan(y/x)}{\sqrt{x^2+y^2}} \\
            f_y(x,y) &= \frac{x}{x^2+y^2} = \frac{\cos(\arctan(y/x))}{\sqrt{x^2+y^2}}\\
            f_{xx}(x,y) &= \frac{2xy}{(x^2+y^2)^2} \\ 
            f_{yy}(x,y) &= -\frac{2xy}{(x^2+y^2)^2}
        \end{align*}
        
        Therefore, maxing the substitutions, \( \Phi_t = \arctan(Y_t/X_t) \), and \( R_t = \sqrt{X_t^2+Y_t^2}  \),
        \begin{align*}
            \d \Phi_t &= - \frac{\sin(\Phi_t)}{R_t}\d W_t^1 + \frac{\cos(\Phi_t)}{R_t}\d W_t^2
        \end{align*}
        
        Similarly,
        \begin{align*}
            \d R_t &=  g_x(X_t,Y_t)\d X_t + g_y(X_t,Y_t)\d Y_t
            \\&\hspace{3em}+ \frac{1}{2}\big(g_{xx}(X_t,Y_t)\d[X,X]_t + g_{xy}(X_t,Y_t)\d [X,Y]_t 
            \\&\hspace{6em}+ g_{yx}(X_t,Y_t)\d[Y,X]_t + g_{yy}(X_t,Y_t)\d[Y,Y]_t\big) 
        \end{align*}
        
        We compute,
        \begin{align*}
            g_x(x,t) &= \frac{x}{\sqrt{x^2+y^2}} = \cos(\arctan(y/x)) \\
            g_y(x,t) &= \frac{y}{\sqrt{x^2+y^2}} = \sin(\arctan(y/x)) \\
            g_{xx}(x,t) &= \frac{y^2}{(x^2+y^2)^{3/2}} \\
            g_{yy}(x,t) &= \frac{x^2}{(x^2+y^2)^{3/2}}
        \end{align*}
        
        Therefore, maxing the substitutions, \( \Phi_t = \arctan(Y_t/X_t) \), and \( R_t = \sqrt{X_t^2+Y_t^2}  \),
        \begin{align*}
            \d R_t &= \cos(\Phi_t)\d W_t^1 + \sin(\Phi_t) \d W_t^2 + \frac{1}{2R_t} \d t
        \end{align*}


    \item
        We know \( u \) satisfies,
        \begin{align*}
            (\cA - \lambda) u = 0, && u(r,0) = u(r,\pi/2) = f(r)
        \end{align*}
        where,
        \begin{align*}
            \cA = \frac{1}{2r} \partial_r + \frac{1}{2}\left(\partial_r + \frac{1}{r^2}\partial_\phi^2 \right)
        \end{align*}
        

        \iffalse
        Define,
        \begin{align*}
            v(r,\phi) = \EE[e^{-\lambda(\tau - t)}f(R_\tau) | X_t=x, \Phi_t =\phi]
        \end{align*}
        
        Define,
        \begin{align*}
            M_t &= e^{-\lambda t} v(R_t,\Phi_t)
            \\&= e^{-\lambda t} \EE[e^{-\lambda (\tau - t)}f(R_\tau) | X_t = x, \Phi_t = \phi]
            \\&= \EE[e^{-\lambda \tau}f(R_\tau) | X_t = x, \Phi_t = \phi]
        \end{align*}
        

        We claim that \( M_t \) is a Martingale. Indeed,
        \begin{align*}
            \EE[ M_t | \cF_s] 
            &= \EE[\EE[ e^{-\lambda \tau} f(R_\tau) | R_t,\Phi_t] | \cF_s]
            \\&= \EE[\EE[ e^{-\lambda \tau} f(R_\tau) | \cF_t] | \cF_s]
            \\&= \EE[ e^{-\lambda \tau} f(R_\tau) | \cF_s]
            \\&= M_s
        \end{align*}
        
        Using our Heuristics \( \d W_t^1 \d W_t^1 = \d W_t^2 \d W_t^2 = \d t \) and \( \d W_t^1 \d W_t^2 = \d t\d W_t^1 = \d t \d W_t^2 = \d t \d t = 0 \) we compute,
        \begin{align*}
            \d[R,R]_t &= 0 \\ %&= \cos^2(\Phi_t) \d t + \sin^2(\Phi_t)\d t = \d t \\
            \d[R,\Phi]_t &=  - \frac{\sin(\Phi_t)\cos(\Phi_t)}{R_t}\d t + \frac{\cos(\Phi_t)\sin(\Phi_t)}{R_t}\d t = 0\\
            \d[\Phi,\Phi]_t &= \frac{\sin^2(\Phi_t)}{R_t^2}\d t + \frac{\cos(\Phi_t)^2}{R_t^2}\d t = \frac{1}{R_t^2}\d t
        \end{align*}
        

        Therefore,
        \begin{align*}
            \d M_t &= \d \left( e^{-\lambda t} v(R_t,\Phi_t) \right) \\
            &= \d \left( e^{-\lambda t} \right) v(R_t,\Phi_t) + e^{-\lambda t} \d v(R_t,\Phi_t)
        \end{align*}
        \fi

\iffalse
        Therefore,
        \begin{align*}
            \d u(R_t,\Phi_t) &= \partial_r u(R_t,\Phi_t) \d R_t + \partial_\phi u(R_t,\Phi_t) \d \Phi_t
            \\& \hspace{3em}+ \frac{1}{2} \big( \partial_r^2 u (R_t,\Phi_t) \d[R,R]_t + 2\partial_{r\phi} u(R_t,\Phi_t) \d[R,\Phi]_t 
            \\&\hspace{6em}+ \partial_{\phi}^2 u(R,\Phi)\d[\Phi,\Phi]_t \big)
            \\&= \left( \frac{1}{2R_t}\partial_r + \frac{1}{2} \frac{1}{R^2}\partial_\phi^2  \right)u(R_t,\Phi_t)\d t + (\cdots)\d W_t^1 + (\cdots)\d W_t^2
        \end{align*}
        
        Since \( u(R_t,\Phi_t) \) is a martingale, the \( \d t \) term must be zero. Therefore,
        \begin{align*}
            \left( \frac{1}{r} \partial_r + \frac{1}{r^2}\partial_\phi^2  \right) u(r,\phi) = 0
        \end{align*}
        
        We have boundary conditions,
        \begin{align*}
            u(r,0) = u(r,\pi/2) = f(r)
        \end{align*}
        \fi 


    \item We can compute \( u(r,\phi) \) by simulating trajectories of \( R \) and \( \Phi \), and observing where they end up. In particular, starting at the point \( (R_0,\Phi_0) = (r,\phi) \), we run a stochastic integrator (forward Euler for instance) until the process exits. We then compute the value of \( e^{-\lambda \tau} f(R_\tau) \). Repeating this integration many times will give an estimate at the expected value.

        To integrate we must set a time step size \( \Delta t \). At each step \( \d W_t^1 \) and \( \d W_t^2 \) will be normally distributed with mean 0 and variane \( \Delta t \). To advance the solution we generate random normal variables with this mean and variance, and then compute the change using our SDEs for \( R \) and \( \Phi \), replacing \( \d t \) with \( \Delta t \) and \( \d W_t^1, \d W_t^2 \) with the random normal variables. This is repeated iteratively.

\end{enumerate}
\end{solution}


\begin{problem}[Practice Exam 2, Problem 1]
Let \( Y = (Y_n)_{n\in\NN_0} \) be a sequence of iid random variables with \( Y_n\sim \operatorname{Pois}(\lambda) \) representing the number of particles entering a chamer at time \( n \). The lifetimes of the particles are iid geometric random variables with parameter \( p \). Let \( X_n \) represent the number of particles in the chamber at time \( n \).
\begin{enumerate}[nolistsep,label=(\alph*)]
    \item Give an expression for \( p(i,j) = \PP(X_{n+1} = j | X_n = i) \).
    \item Find the stationary distribution \( \pi \) of \( X = (X_n)_{n\geq 0} \).
\end{enumerate}
\end{problem}

\begin{solution}[Solution]
\begin{enumerate}[label=(\alph*)]
    \item
        If the lifetime of a particle is a geometric random variable with parameter \( p \), then at each step there is a probability \( p \) that the particle will decay and a probability \( 1-p \) that the particle will not decay.

        Let \( Z_n \) represent the number of particles which no \textit{not} decay during the \( n \)-th step. That is,
        \begin{align*}
            X_{n+1} = Z_n + Y_n 
        \end{align*}
        
        Since each of the \( X_n \) particles no not decay with probability \( 1-p \) we have \( Z_n \sim \operatorname{Bin}(X_n,1-p) \) and \( Y_n \sim \operatorname{Pois}(\lambda) \).

        Denote the generating functions of \( Y_n \) and \( Z_n \) by, \( G_{Y_n}(s) \) and \( G_{Z_n}(s) \) respectively.
        Explicitly,
        \begin{align*}
            G_{Y_n}(s) = G_Y(s) = e^{\lambda(s-1)}, &&
            G_{Z_n}(s) = (p+(1-p)s)^{X_n} = G_{X_n}(p+(1-p)s)
        \end{align*}
        
        Assume \( Y_n \) is independent of \( X_n \) and therefore of \( Z_n \). 
        If \( X_n = i \) the generating function is \( G_{X_n} = s^i \). We can then write,
        \begin{align*}
            G_{X_{n+1}}(s) = G_{Y_n}(s)G_{Z_n}(s) = G_Y(s) (p+(1-p)s)^i = e^{\lambda(s-1)} (p+(1-p)s)^i
        \end{align*}

        Therefore,
        \begin{align*}
            p(i,j) &= \PP(X_{n+1} = j | X_n = i) 
            \\&= \left[\frac{1}{j!} \dd[j]{G_{X_{n+1}}(s)}{s} \right]_{s=0}
            \\&= \left[\frac{1}{j!} \dd[j]{}{s} \left[ e^{\lambda(s-1)}(p+(1-p)s)^i \right] \right]_{s=0}
        \end{align*}

       
    \item 
        More generally, the generating function \( G_{X_{n+1}}(s) \) of \( X_{n+1} \) is then,
        \begin{align*}
            G_{X_{n+1}}(s) = G_{Y_n}(s)G_{Z_n}(s) = G_{Y}(s)G_{X_{n}}(p+(1-p)s) 
        \end{align*}
        
        This gives a recurrence relation. We assume \( X_0 = 0 \) so that \( G_{X_0}(s) = 1 \).
        For convencience write \( q = 1-p \). Then,
        \begin{align*}
            1+q^k(s-1) |_{s=(1+q(s-1))} = 1+q^k((1+q(s-1))-1) = 1+q^{k+1}(s-1)
        \end{align*}
        
        Therefore,
        \begin{align*}
            G_{X_n}(s) &= G_Y(s) G_{X_{n-1}}(1+q(s-1)) \\
            &= G_Y(s) G_Y(1+q(s-1)) G_{x_{n-2}}(1+q^2(s-1)) \\
            &\hspace{.6em}\vdots \\
            &= \prod_{k=0}^n G_Y(1+q^k(s-1))
        \end{align*}
       
        We can rewrite this as,
        \begin{align*}
            G_{X_n}(s) = \exp \left( \sum_{k=0}^n \lambda((1+q^k(s-1))-1) \right)
            = \exp \left( \lambda(s-1) \sum_{k=0}^n q^k \right)
        \end{align*}
        
        Taking the limit as \( n\to\infty \) we find,
        \begin{align*}
            G_{X_{\infty}}(s) 
            &= \lim_{n\to\infty} G_{X_n}(s) 
            \\&= \exp \left( \lambda(s-1) \sum_{k=0}^{\infty} q^k \right) 
            \\&= \exp \left( \frac{\lambda(s-1)}{1-q}  \right) 
            \\&= \exp\left( \frac{\lambda}{p}(s-1)\right)        
        \end{align*}
        
        Therefore, by the continuity theorem, \( X_\infty \) is distributed like a Poisson random variable with parameter \( \lambda/p \). 

        This means the invariant distribution of \( X \) is the density function of a Poisson random variable with parameter \( \lambda/p \). That is,
        \begin{align*}
            \pi(k) = \left( \frac{\lambda}{p} \right)^k \frac{e^{-\lambda/p}}{k!}
        \end{align*}
        
\end{enumerate}
\end{solution}


\begin{problem}[Practice Exam 2, Problem 2]
    Fix a probability space \( (\Omega, \cF, \PP) \) and a filtration \( \bF = (\cF_t)_{0\leq t\leq T} \) where \( T< \infty \). Consider a process \( P = (P_t)_{0\leq t\leq T} \) defined as,
    \begin{align*}
        P_t = \EE[\bOne_{X_T\leq a} | \cF_t], && \d X_t = \d W_t
    \end{align*}
    where \( W \) is a \( (\PP,\bF) \)-Brownian motion. Derive an SDE for the process \( P \). Your answer should not involve \( X \). You may find it useful to use the CDF \( \Phi \) of a standard normal random variable,
    \begin{align*}
        \Phi(z) = \frac{1}{\sqrt{2\pi}} \int_{-\infty}^{z} e^{-x^2/2}\d x
    \end{align*}
    its inverse \( \Phi^{-1} \) and its derivative \( \phi:= \Phi' \).
\end{problem}

\begin{solution}[Solution]
Note that condition on \( \cF_t \) is the same as conditioning on \( X_t \). For notational convenience let \( x=X_t \). Then,
\begin{align*}
    P_t = \EE[\bOne_{X_T\leq a} | \cF_t] = \PP(X_T\leq a | X_t = x)
    = \int_{-\infty}^a \Gamma(t,x;T;y)\d y
\end{align*}

Note further that,
\begin{align*}
    \Gamma(t,x;T,y) = \frac{1}{\sqrt{2\pi (T-t)}}\exp \left( -\frac{(y-x)^2}{2(T-t)} \right) 
    = \frac{1}{\sqrt{T-t}}\phi \left( \frac{y-x}{\sqrt{T-t}} \right)
\end{align*}

    Let \( u = (y-x)/\sqrt{T-t} \). Then \( \d y = \sqrt{T-t} \d u \) so,
    \begin{align*}
       P_t
        = \int_{-\infty}^{a}  \frac{1}{\sqrt{T-t}}\phi \left( \frac{y-x}{\sqrt{T-t}} \right)\d y
        = \int_{-\infty}^{\frac{a-x}{\sqrt{T-t}}} \phi(u)\d u
        = \Phi \left( \frac{a-x}{\sqrt{T-t}} \right)
    \end{align*}
    
Recall that \( X_t = x \) so,
\begin{align*}
    P_t = \Phi \left( \frac{a-X_t}{\sqrt{T-t}} \right)
   \end{align*}


    Let \( Y_t = (a-X_t)/\sqrt{T-t} = \Phi^{-1}(P_t) \). Then,
    \begin{align*}
        \d Y_t = \frac{1}{2(T-t)} \frac{a-X_t}{\sqrt{T-t}}\d t-\frac{1}{\sqrt{T-t}}\d X_t 
        = \frac{Y_t}{2(T-t)} \d t - \frac{1}{\sqrt{T-t}}\d W_t
    \end{align*}
    
    Therefore,
    \begin{align*}
        \d P_t &= \d \Phi(Y_t) = \phi(Y_t)\d Y_t + \frac{1}{2} \phi'(Y_t)\d[Y,Y]_t 
        \\&= \phi(Y_t) \frac{Y_t}{2(T-t)}\d t - \phi(Y_t) \frac{1}{\sqrt{T-t}}\d W_t + \frac{1}{2}\phi'(Y_t) \frac{1}{T-t}\d t
        \\&= \left(\phi(Y_t) \frac{Y_t}{2(T-t)} + \frac{1}{2}\phi'(Y_t) \frac{1}{T-t} \right)\d t  - \frac{\phi(Y_t)}{\sqrt{T-t}}\d W_t 
    \end{align*}
    We know that \( \phi'(z) = z\phi(z) \).
    (alternative argument: As \( P_t \) is a martingale we know the \( \d t \) term is zero.) 
    Therefore,
    \begin{align*}
        \d P_t = -\frac{\phi(\Phi^{-1}(P_t))}{\sqrt{T-t}}\d W_t
    \end{align*}
    


\end{solution}



\begin{problem}[Practice Exam 3, Problem 1]
Let \( X = (X_t)_{t\geq 0} \) be a continuous time Markov chain with \( X_t \) representing the number of individuals in a population at time \( t \). Individuals do not reproduce. However, immigrants join the population as a Poisson process with parameter \( \lambda \). The lifetimes of individuals are iid exponentially distributed random varaibles with parameter \( \mu \).
\begin{enumerate}[nolistsep,label=(\alph*)]
    \item Write the generator \( G \) of \( X \)
    \item Find the stationary distribution \( \pi \) of \( X \).
\end{enumerate}
\end{problem}

\begin{solution}[Solution]
\begin{enumerate}[label=(\alph*)]
    \item
        This is a M/M/\(\infty\) queue with arrival parameter \( \lambda \) and service parameter \( \mu \).
        
        In a short time \( s\) the probability of no immigrant joining the population is \( 1-\lambda s + \cO(s^2) \), the probability of one immigrant joining is \( \lambda s + \cO(s^2) \), and the probability more than one immigrant joining is \( \cO(s^2) \).

        Similarly, the probability of a given individual dying is \( \mu s + \cO(s^2) \) and the probability of an individual not dying is \( 1-\mu s + \cO(s^2) \). 

        The generator is then,
        \begin{align*}
            G = \left[\begin{array}{ccccccc}
            -\lambda & \lambda \\
            \mu & -(\mu+\lambda) & \lambda \\
            & 2\mu & - (2\mu+\lambda) & \lambda \\
            && 3\mu & -(3\mu+\lambda) & \lambda \\
                &&&\ddots&\ddots&\ddots 
            \end{array}\right]
        \end{align*}
       
       \item 
        An M/M/\(\infty\) queue with these parameters has invariant distribution,
        \begin{align*}
            \pi(k) = \frac{(\lambda/\mu)^k e^{-\lambda\mu}}{k!}
        \end{align*}
\end{enumerate}
\end{solution}


\begin{problem}[Practice Exam 3, Problem 2]
Fix a probability space \( (\Omega, \cF,\PP) \) and a filtration \( \bF = (\cF_t)_{t\geq 0} \). Consider a process \( X = (X_t)_{t\geq 0} \) that satisfies the following SDE,
\begin{align*}
    \d X_t = b\d t + a\d W_t
\end{align*}
where \( W \) is a \( (\PP,\bF) \)-Brownian motion. Suppose that \( X_0 \in (L,R) \) and that \( \{L\} \) and \( \{R\} \) are reflecting boundaries.
\begin{enumerate}[nolistsep,label=(\alph*)]
    \item Derive an expression for the invariant distribution of \( X \).
    \item Derive an expression for the transition density \( \Gamma(t,x;T,y)\d y:= \PP(X_t \in\d y | X_t = x) \).
    \item Show that \( \Gamma(t,x;T,y) \to \pi(y) \) as \( T\to\infty \).
\end{enumerate}
\end{problem}

\begin{solution}[Solution]
\begin{enumerate}[label=(\alph*)]
    \item We have scale and speed densities,
        \begin{align*}
            s(x) &= \exp \left( -2\int_L^R \frac{b}{a^2}\d x \right)
            = \exp \left( -2 \frac{b}{a^2} (R-L) \right)
            \\
            m(x) &= \frac{1}{a^2}\exp \left( 2\int_L^R \frac{b}{\sigma^2(x)}\d x \right)
            = \frac{1}{a^2} \exp \left( 2 \frac{b}{a^2} (R-L) \right)
        \end{align*}

        Since \( m(x) \) is constant on a finite interval it can me normalized (for \( a\neq 0 \)) as \( m(x) = 1/(R-L) \), which is therefore the invariant distribution of \( X \). 

        This makes sense since as time evolves, the Brownian motion term will dominate, and the reflecting boundary condition means that the process becomes constant.

        
        
    \item We have infinitesimal generator,
        \begin{align*}
            \cA(t) = b\partial_x + \frac{1}{2}a^2\partial_x^2 
        \end{align*}

        The boundary conditions require \( \cA \) act on functions satisfying,
        \begin{align*}
            \left[\frac{1}{s(x)}\partial_x f(x) \right]_{x\in\{L,R\}} = 0
        \end{align*}
        \note{I dont quite understand this}

        We seek a complete set of eigenfunctions \( \{\psi_n(x)\}_{n=0}^{\infty} \) with corredponing eignevlaues \( \lambda_n \) of \( \cA \) satisfying this boundary condition and normalized with respect to \( m \). In this case,
        \begin{align*}
            \Gamma(t,x;T,y) = m(y) \sum_{n=0}^{\infty} e^{(T-t) \lambda_n}\psi_n(y)\psi_n(x)
        \end{align*}
       

        We now find such functions:
       
        \note{I have no idea}
       
    \item
        \note{Presumably this is easy if you can do (b)}
\end{enumerate}
\end{solution}

\begin{problem}[Practice Exam 4, Problem 1]
    A transition probablity matrix \( P \) for a Markov chain with \( N \) states is said be doubly stochastic if the entries in each of its columns add up to one.
\begin{enumerate}[nolistsep,label=(\alph*)]
\item Show that the uniform distribution given by \( q_i = 1/N \) for all \( j \) is a stationary distribution for such a Markov chain.
\item Consider the following random walk on the sets of integers \( \{0, 1, \ldots, L\} \). The walk jumps to the right or left at each step with probability 1/2 subject to the rule that if it tries to go to the left from 0 or to the right from \( L \) it stays put. Compute the stationary distribution of this random walk.
\item Consider the following random walk on state-space \( \{0, 1, 2, \ldots, L\} \) of numbers arranged on a ring. At each step, the walk goes to the right with probability \( a \) or to the left with probability \( 1 - a \) subject to the rules if it tries to go to the left from 0 it ends up at \( L \) or if it tries to go to the right from \( L \) it ends up at 0. Compute the stationary distribution of this chain.
\end{enumerate}
\end{problem}

\begin{solution}[Solution]
\begin{enumerate}[label=(\alph*)]
    \item By definition, since \( P \) is doubly stochastic, \( \sum_i P_{ij} = 1 \) for all \( j \). Trivially,
        \begin{align*}
            (qP)_j = \sum_{i=1}^N q_{i}P_{ij} = \sum_{i=1}^N \frac{1}{N} P_{ij} = \frac{1}{N}\sum_{i=1}^N P_{ij} = \frac{1}{N} = q_i
        \end{align*}

        This proves \( qP = q \). That is, \( q \) is a stationary distribution of \( P \).

    \item We have probability transition matrix,
        \begin{align*}
            P = \left[\begin{array}{cccccc}
                1/2 & 1/2 \\
                1/2 & &1/2 \\
                &1/2 & &1/2 \\
                &&1/2 &  &\ddots \\
                &&&\ddots &&1/2 \\ 
                &&&&1/2 & 1/2
            \end{array}\right]
        \end{align*}
       
        This is doubly stochastic so it has an invariant distribution,
        \begin{align*}
            \pi = [1/L,1/L,\ldots, 1/L]
        \end{align*}

        This is a finite irreducible chain so the stationary distribution is unique.

        
    \item We have probability transition matrix,
        \begin{align*}
            P = \left[\begin{array}{cccccc}
                & 1/2 &&&& 1/2 \\
                1/2 & &1/2 \\
                &1/2 & &1/2 \\
                &&1/2 &  &\ddots \\
                &&&\ddots &&1/2 \\ 
                1/2 &&&&1/2 & 
            \end{array}\right]
        \end{align*}
       
        This is doubly stochastic so it has an invariant distribution,
        \begin{align*}
            \pi = [1/L,1/L,\ldots, 1/L]
        \end{align*}
        
        This is a finite irreducible chain so the stationary distribution is unique.

\end{enumerate}
\end{solution}


\begin{problem}[Practice Exam 4, Problem 2]
    Fix a probability space \( (\Omega, \cF, \PP\) and a filtration \( \bF = (\cF_t )_{0\leq t \leq T} \). Suppose that \( X = (X_t)_{t\geq 0} \) satisfies the following SDE,
    \begin{align*}
        \d X_t = - \kappa X_t\d t + \sigma \d W_t
    \end{align*}
    where \( W \) is a \( (\PP,\bF) \)-Brownian motion. Now consider a chance of measure,
    \begin{align*}
        \dd{\tilde{\PP}}{\PP} = \exp \left( -\frac{1}{2}\gamma^2 T - \gamma W_T \right).
    \end{align*}
\begin{enumerate}[nolistsep,label=(\alph*)]
    \item Derive an SDE for the process \( X \) under \( \tilde{\PP} \). Your answer should be given in terms of a process \( \tilde{W} \) which is a \( (\tilde{\PP},\bF) \)-Brownian motion.
    \item Define functions \( u:[0,T]\times \RR \to \RR \) and \( \tilde{u}:[0,T]\times \RR \to \RR \) as follows
        \begin{align*}
            u(t,x):= \EE[h(X_T) | X_t = x], && \tilde{u}(t,x):= \tilde{\EE}[h(X_T) | X_t = x]
        \end{align*}
        Provide the PDEs satisfied by \( u \) and \( \tilde{u} \), respectively.
\end{enumerate}


\end{problem}

\begin{solution}[Solution]
\begin{enumerate}[label=(\alph*)]
    \item By Girsanov's theorem we know that the process \( \tilde{W} \) defined as,
        \begin{align*}
            \d \tilde{W}_t = \gamma \d t + \d W_t
        \end{align*}
        is a Brownian motion under \( \tilde{\PP} \). Therefore, writing \( X \) in terms of this process we have,
        \begin{align*}
            \d X_t = - \kappa X_t \d t + \sigma \left( \d \tilde{W}_t - \gamma \d t \right) 
            = - \left( \sigma \gamma + \kappa X_t \right) \d t + \sigma \d \tilde{W}_t
        \end{align*}
        
    \item
        We have that \( u \) and \( \tilde{u} \) satisfy the KBE. In particular,
        \begin{align*}
            \left[ \left( \partial_t - \kappa \partial_x + \frac{1}{2}\sigma^2 \partial_x^2 \right)u(t,x)\right]_{x=X_t} =0 , && u(T,x) = h(x)
        \end{align*}
        
        \begin{align*}
            \left[ \left( \partial_t - (\sigma \gamma + \kappa x) \partial_x + \frac{1}{2}\sigma^2 \partial_x^2 \right) \tilde{u}(t,x) \right]_{x=X_t} =0 , && \tilde{u}(T,x) = h(x)
        \end{align*}
        



\end{enumerate}
\end{solution}

\begin{problem}[Practice Exam 5, Problem 1]
    Consider a Markov chain with state space \( \{0,1,2,\ldots\} \) and transition probabilities,
    \begin{align*}
        p(i,i+1) &= p_i, && i\geq 0 \\
        p(i,i-1) &= q_i, && i> 0 \\
        p(i,i) &= r_i, && i\geq 0
    \end{align*}
    For \( N > 0 \) and state \( i \), let \( a_N(i) \) be the probability that the time of first visit to state \( N \) is strictly less than the time of first visit to state \( 0 \) if we start at state \( i \). Note that \( a_N(0) = 0 \) and \( a_N(N) = 1 \).
\begin{enumerate}[nolistsep,label=(\alph*)]
    \item Write a recursive relation for \( a_N(i) \) by consider what happens on the first transition out of state \( i \).
    \item Solve the above equation to compute \( a_N(i) \).
    \item Use (b) above to show that state 0 is recurrent if and only if,
        \begin{align*}
            \sum_{j=1}^{\infty} \prod_{i=1}^{j-1} \frac{q_i}{p_i} = \infty
        \end{align*}
    \item Analyze the situation where \( p_i = p \), \( q_i = 1-p \), \( r_i = 0 \) for \( x \geq 1 \), and \( r_0 = 1-p \).
\end{enumerate}

\end{problem}

\begin{solution}[Solution]
\begin{enumerate}[label=(\alph*)]
    \item Fix a state \( i\in\ZZ_{+} \). If we transition down when we first leave \( i \), the probability of reaching state \( N \) before state 0 is \( a_N(i-1) \). Similarly, if we transition up when we first leave \( i \), the probability of reaching state \( N \) before state 0 is \( a_N(i+1) \). 
        
    The probability of transitioning up when leaving state \( i \) is \( p_i / (p_i+q_i) \) and the probability of transitioning down when leaving state \( i \) is \( q_i / (p_i-q_i) \). We therefore have the relationship,
    \begin{align*}
        a_N(i) = \frac{p_i}{p_i+q_i} a_N(i+1) + \frac{q_i}{p_i+q_i}a_N(i-1)
    \end{align*}

    \item We solve this using Mathematica.
    \begin{lstlisting}
        RSolve[{a[i] == ((p[i - 1] + q[i - 1]) a[i - 1] - q[i - 1] a[i - 2])/p[i - 1], a[0] == 0, a[n] == 1}, a[i], i]
    \end{lstlisting}
    

        This yields solution,
        \begin{align*}
            a_N(i) = \frac{\sum_{k=1}^{i}\prod_{j=1}^{k-1} \frac{q_{j}}{p_{j}}}{\sum_{k=1}^{N}\prod_{j=1}^{k-1} \frac{q_{j}}{p_{j}}}
        \end{align*}
        

    \item

        \iffalse
        By definition, state \( 0 \) is recurrent if,
        \begin{align*}
            \PP(X_n = 0\text{ for some } n | X_0 = 0) = 1 
        \end{align*}
        
        More precisely,
        \begin{align*}
            \lim_{N\to\infty} \PP( \exists~ n < N: X_n = 0 | X_0=0) = 1
        \end{align*}
       
        Taking the compliment,
        \begin{align*}
            \lim_{N\to\infty} \PP( \forall n < N, X_n \neq 0 | X_0 =0 ) = 0
        \end{align*}
       
        \note{PROBABLY NEED MORE ARGUMENT HERE} also what \( N \) means has changed.
        (I THINK) equivalently,
        \begin{align*}
            \lim_{N\to\infty} a_N(i) = \lim_{N\to\infty} \PP(\tau_N < \tau_0|X_0=i) = 0
        \end{align*}
        
        This happens if and only if the denominator of \( a_N(i) \) is infinite. That is,
        \begin{align*}
            \lim_{N \to\infty} \sum_{k=1}^{N} \prod_{j=1}^{k-1} \frac{q_j}{p_j} 
            = \sum_{k=1}^{\infty} \prod_{j-1}^{k-1} \frac{q_j}{p_j}
            = \infty
        \end{align*}
       
        \hrulefill
        \fi

        Define,
        \begin{align*}
            \tau_0 = \operatorname{inf}\{ t > 0 : X_t = 0 \},&&
            \tau_N = \operatorname{inf}\{ t > 0 : X_t = N \}
        \end{align*}
        
        By definition, state 0 is recurrent if and only if,
        \begin{align*}
            \PP(\tau_0 < \infty | X_0 = 0 ) = \lim_{T\to\infty} \PP(\tau_0 \leq T | X_0 = 0) = 1
        \end{align*}
        
        Since, given \( X_0 = 0 \),
        \begin{align*}
            \lim_{N\to\infty} \tau_N = \infty
        \end{align*}
        
        State 0 is recurrent if and only if,
        \begin{align*}
            \lim_{N\to\infty} \PP(\tau_0 \leq \tau_N | X_0 = 0) = 1
        \end{align*}
        
        Taking the compliment we have that state 0 is recurrent if and only if,
        \begin{align*}
            \lim_{N\to\infty} \PP(\tau_N < \tau_0 | X_0 = 0) = 0
        \end{align*}
        

        In order to reach state \( N \) before returning to state \( 0 \), we must first move to state 1, and then reach state \( N \) before reaching state \( 0 \). We move from state 0 to state \( N \) with probability \( p_0 \), and we reach state \( N \) before state \( 0 \) from state 1 with probability \( a_N(1) \). Therefore,
        \begin{align*}
            \PP(\tau_N < \tau_0 | X_0=0) = \PP(\tau_N < \tau_0) = p_0a_N(1)
        \end{align*}

        We then have that state 0 is recurrent if and only if,
        \begin{align*}
            \lim_{N\to\infty} p_0a_N(1) = 0 
        \end{align*}
        
\iffalse
        Therefore state 0 is recurrent if and only if,
        \begin{align*}
            p_0 = 0 \text{ or } \lim_{N\to\infty} a_N(1) = 0
        \end{align*}
\fi
        Finally, we see that \( \lim_{N\to\infty} a_N(1) = 0 \) if and only if,
        \begin{align*}
            \frac{1}{p_0}\sum_{k=1}^\infty \prod_{j=1}^{k-1}\frac{q_j}{p_j}
            = \lim_{N\to\infty}\frac{1}{p_0}\sum_{k=1}^N \prod_{j=1}^{k-1}\frac{q_j}{p_j}
            = \infty
        \end{align*}

    
    \item
        Let \( s = q/p \). Then,
        \begin{align*}
            \frac{1}{p_0}\sum_{j=1}^{\infty} \prod_{i=1}^{j-1} \frac{q_i}{p_1} = \frac{1}{p} \sum_{j=1}^{\infty} s^{j-1} = \sum_{j=0}^{\infty} s^j
        \end{align*}
        
        This is convergent if and only if \( q/p = s < 1 \). Therefore state 0 is recurrent if and only if \( q \geq p \) or \( p=0 \).


\end{enumerate}
\end{solution}


\begin{problem}[Practice Exam 5, Problem 2]
    Fix a probability space \( (\Omega,\cF,\PP) \) and a filtration \( \bF = (\cF_t)_{t\geq 0} \). Consider a mean-repelling OU process,
    \begin{align*}
        \d X_t = X_t\d t + \sqrt{2} \d W_t
    \end{align*}
    where \( W \) is a \( (\PP,\bF) \)-Brownian motion.
\begin{enumerate}[nolistsep,label=(\alph*)]
    \item Derive two representations of the transition density \( \Gamma(t,x,T,y)\d y := \PP(X_t \in\d y | X_t = x) \) of \( X \). One of the representations should involve Hermite polynomials.
    \item Does the process \( X \) have an invariant distribution? If so, provide it. If not, explain why not.
\end{enumerate}
\end{problem}

\begin{solution}[Solution]
\begin{enumerate}[label=(\alph*)]
    \item Recall that \( X_t \) can be written explicitly in terms of \( W_t \) as,
        \begin{align*}
            X_T = e^{T-t} X_t + \int_t^T \sqrt{2}e^{T-s} \d W_s %= e^{T-t} X_t + \sqrt{2} e^T \int_t^T e^{-s}\d W_s
        \end{align*}

        
        Recall that \( \int_t^T g(s)\d W_s \) is normally distributed with mean zero and variance \( \int_t^T g(s)^2 \d s \).
        Therefore \( X_T \) is distributed normally with mean \( m(T,t,X_t) = e^{T-t}X_t \) and variance,
        \begin{align*}
            v(T,t,X_t) = \int_t^T \left( \sqrt{2} e^{T-s} \right)^2\d s = e^{2(T-t)}-1
        \end{align*}
        

        Therefore,
        \begin{align*}
            \Gamma(t,x,T,y)\d y = \PP(X_t\in\d y | X_t = x) = \phi(y)\d y
        \end{align*}
        where \( \phi(y) \) is the density of a normal random variable with the above mean and variance \( m(T,t,x) \) and \( v(T,t,x) \) respectively.
        
        Explicitly,
        \begin{align*}
            \Gamma(t,x,T,y)\d y = \frac{\d y}{ \sqrt{2\pi\left( e^{2(T-t)}-1 \right)}} \exp \left( - \frac{1}{2} \frac{ (y-xe^{T-t})^2}{ e^{2(T-t)}-1}  \right)
        \end{align*}
        
        \hrulefill

        We now compute \( \Gamma(t,x,T,y) \) by solving the KBE. In particular we have,
        \begin{align*}
            (\partial_t + \cA) \Gamma(t,x,T,y) = 0, && \Gamma(T,x,T,y) = \delta_y, && \cA = x\partial_x + \partial_x^2
        \end{align*}

        Write speed density,
        \begin{align*}
            m(x) = \exp \left( \int x \d x \right) = \exp(x^2/2)
        \end{align*}
        

        Suppose we have a complete set of eigenfunctions \( \psi_n \) of \( \cA \) satisfying \( \cA \psi_n = \lambda_n\psi_n \) normalized with respect to the speed density \( m(x) \).

        In this case,
        \begin{align*}
            \Gamma(t,x,T,y) = m(y) \sum_{n} e^{(T-t)\lambda_n}\psi_n(y) \psi_n(x)
        \end{align*}
        
        We now funch such eigenfunctions. Indeed, 
        
        \note{Use forward equation to find eigenfunctions, then convert to eigenfunctions of backward equation}

        Define,
        \begin{align*}
            \hat{\psi}_n(x) = e^{-x^2/2}H_n(x) %= e^{-x^2/2} (-1)^n e^{x^2/2} \dd[n]{}{x}e^{-x^2/2} = (-1)^n \dd[n]{}{x} e^{-x^2/2}
        \end{align*}

        The Hermite polynomials have the properties,
        \begin{align*}
            H_{n+1} = xH_n - H_n', && H_n' = nH_{n-1}
        \end{align*}
        
        Therefore,
        \begin{align*}
            \hat{\psi}_n' &= -xe^{-x^2/2}H_n + e^{-x^2/2}H_n' = (H_n'-xH_n)e^{-x^2/2} = -H_{n+1}e^{-x^2/2} = -\hat{\psi}_{n+1} \\
        \end{align*}
       
        We then have,
        \begin{align*}
            x\hat{\psi}_n' &= -xH_{n+1}e^{-x^2/2} = -(H_{n+2}-H_{n+1}')e^{-x^2/2} = -\hat{\psi}_{n+2} + (n+1) \hat{\psi}_n \\
            \hat{\psi}_n''&= \hat{\psi}_{n+2}
        \end{align*}
        
        Then,
        \begin{align*}
            \cA \hat{\psi}_n = (n+1)\hat{\psi}_n
        \end{align*}
        
        Now define,
        \begin{align*}
            \psi_n = \frac{\hat{\psi}_n}{\lVert\hat{\psi}_n\rVert_m} 
        \end{align*}
       
        Finally \note{argument about completeness}


 
    \item No.

        Argument about spreading out.

\end{enumerate}
\end{solution}


\begin{problem}[Practice Exam 6, Problem 1]
Consider a game which is played on a network with nodes numbered \( 1, 2, 3, 4 \) and edges that connect these nodes as show below. Note that each node in this network has self-loop edges. Every edge in the network has an associated weight. For example, the weight associated with the edge connecting nodes 1 and 2 is 3. Similarly the weight associated with the self-loop edge on node 1 is 1. The game proceeds as follows. You have a token that you move randomly from one node to another with probabilities propotional to the corresponding edge weights. For example, the probability that your token moves from node 1 to node 2 is \( 3/(1 + 2 + 3) = 1/2 \). Let \( X_n \) be the position of your token after \( n \) moves, for \( n = 0, 1, 2, \ldots \).
\begin{center}
\begin{tikzpicture}[shorten >=1pt,auto,node distance=3cm,
  thick,main node/.style={circle,draw,minimum size=7mm}]

  \node[main node] (1) {1};
    \node[main node] (2) [right of=1] {2};
  \node[main node] (3) [below of=2] {3};
  \node[main node] (4) [below of=1] {4};

  \path[every node/.style={fill=white,inner sep=1pt}, every loop/.style={}]
    (1) edge[in=90,out=180, loop] node {1} (1)
        edge node {3} (2)
    (2) edge[in=0, out=90, loop] node {1} (2)
        edge node {4} (3)
    (3) edge[in=270, out=0, loop] node{1} (3)
        edge node {5} (4)
    (4) edge[in=270, out=180, loop] node{1} (4)
        edge node {2} (1)
        ;
\end{tikzpicture}
\end{center}
\begin{enumerate}[nolistsep,label=(\alph*)]
    \item Model the stochastic process \( X_n \) as a Markov chain by drawing its state transition diagram and write the corresponding one-step transition probability matrix.
    \item Calculate the stationary distribution that your token is at node \( j \), for \( j = 1, 2, 3, 4 \).
    \item Find the expected number of token moves between two consecutive visits to node 2.
\end{enumerate}
\end{problem}

\begin{solution}[Solution]
\begin{enumerate}[label=(\alph*)]
    \item We have transition diagram, 
    \begin{center}
    \begin{tikzpicture} 

    \draw node[circle, draw, minimum size=7mm] (1) at (0,3) {1};
    \draw node[circle, draw, minimum size=7mm] (2) at (3,3) {2};
    \draw node[circle, draw, minimum size=7mm] (3) at (3,0) {3};
    \draw node[circle, draw, minimum size=7mm] (4) at (0,0) {4};


        \path[->,every node/.style={fill=white,inner sep=1pt}, every loop/.style={->}]
    (1) edge[in=90,out=180, loop] node {1/6} (1)
        edge[bend left=30] node {1/2} (2)
        edge[bend left=30] node {1/3} (4)
    (2) edge[in=0, out=90, loop] node {1/8} (2)
        edge[bend left=30] node {1/2} (3)
        edge[bend left=30] node {3/8} (1)
    (3) edge[in=270, out=0, loop] node{1/10} (3)
        edge[bend left=30] node {1/2} (4)
        edge[bend left=30] node {2/5} (2)
    (4) edge[in=270, out=180, loop] node{1/8} (4)
        edge[bend left=30] node {1/4} (1)
        edge[bend left=30] node {5/8} (3)
        ;
    \end{tikzpicture}
    \end{center}

        The corresponding probability transition matrix \( P \) is,
        \begin{align*}
            P = \left[\begin{array}{cccc}
                1/6 & 1/2 & 0 & 1/3 \\
                3/8 & 1/8 & 1/2 & 0 \\
                0 & 2/5 & 1/10 & 1/2 \\
                1/4 & 0 & 5/8 & 1/8
            \end{array}\right]
        \end{align*} 
    \item 
        We easily compute the right eigenvector corresponding to eigenvalue \( 1 \) as,
        \begin{align*}
            \pi = \frac{1}{4}[3/4,1,5/4,1] = [3/16,1/4,5/16,1/4]
        \end{align*}
        
    \item The expected number of moves is the mean recurrence time. This chain is irreducible so by theorem we have,
        \begin{align*}
            \tau_2 = 1/\pi(2) =4
        \end{align*}
        
\end{enumerate}
\end{solution}


\begin{problem}[Practice Exam 6, Problem 2]
    Fix a probability space \( (\Omega,\cF,\PP) \) and a filtration \( \bF = (\cF_t)_{t\geq 0} \). Let \( X \) be an OU process, and \( S \) a strictly increasing L\'evy process (also known as a subordinator), whose dynamics are given by,
    \begin{align*}
        \d X_t = - X_t \d t + \sqrt{2}\d W_t, && \d S_t = \gamma \d t + \int_0^\infty z N(\d t,\d z), && S_0 = 0
    \end{align*}
    where \( W \) is a \( (\PP,\bF) \)-Brownian motion and \( N \) is a poisson random measure with associated L\'evy measure \( \nu \). Consider a Subordianted OU process \( Y \) defined as follows,
    \begin{align*}
        Y_t = X_{S_t}
    \end{align*}
    Define the two-parameter semigroup \( \cP(t,T) \) associated with the \( Y \) process,
    \begin{align*}
        \cP(t,T) f(y) := \EE[f(Y_T) | Y_t = y]
    \end{align*}
    For a fixed \( 0\leq t\leq T <\infty \), what are the eigenfunctions and associated eigenvalues of the operator \( \cP(t,T) \)?
\end{problem}

\begin{solution}[Solution]
\begin{enumerate}[label=(\alph*)]
    \item 
    \item 
\end{enumerate}
\end{solution}


\begin{problem}[Practice Exam 7, Problem 1]
\begin{enumerate}[nolistsep,label=(\alph*)]
    \item (Weather chain) The weather can be either sunny, smoggy, or rainy. The weather stays sunny for an exponentially distributed amount of time with mean 3 days and then turns smoggy. It stays smoggy for an exponentially distributed amount of time with mean 4 days and then turns rainy. Finally, it rains for an exponentially distributed amount of time with mean 1 and then it is sunny. Model the weather system as a continuous time Markov chain and compute its stationary distribution.
    \item (Barbershop) Consider a barbershop with one barber and two waiting chairs. The barber cuts hair at rate 3 customers per hour (exponentially distributed hair-cutting time). Customers arrive according to a Poisson process with rate 2 per hour. Arriving customers leave immediately if they find that the two waiting chairs are occupied. Model this system as a continuous time Markov chain and derive its stationary distribution.
\end{enumerate}
\end{problem}

\begin{solution}[Solution]
\begin{enumerate}[label=(\alph*)]
    \item Let the state space be \( \{1,2,3\} \) where 1 means sunny, 2 means smoggy, and 3 means rainy.
        By definition of the generator of the Markov process we have generator,
        \begin{align*}
            G = \left[\begin{array}{rrr}
            -1/3 & 1/3 & 0 \\
                0 & -1/4 & 1/4 \\
            1 & 0 & -1
            \end{array}\right]
        \end{align*}
        
        We can easily verify that \( \pi G = 0 \) if,
        \begin{align*}
            \pi = [3/8,1/2,1/8]
        \end{align*} 

    \item 
        Let the state space \( \{0,1,2,3\} \) denote the number of people in the queue. \note{does 2 waiting chairs mean there is also a cutting chair?}
        By definition of the generator of a Markov process we have generator,
        \begin{align*}
            G= \left[\begin{array}{rrrr}
                -2 & 2 \\
                3 & -(3+2) & 2 \\
                & 3 & -(3+2) & 2 \\
                & & 3 & -3
            \end{array}\right]
        \end{align*}

        We can easily verify that \( \pi G = 0 \) if,
        \begin{align*}
            \pi = [1/4,1/4,1/4,1/4]
        \end{align*}     


\end{enumerate}
\end{solution}


\begin{problem}[Practice Exam 7, Problem 2]
    Fix a probability space \( (\Omega,\cF,\PP) \) and a filtration \( \bF = (\cF_t)_{0\leq t\leq T} \). Consider two processes,
    \begin{align*}
        \d X_t = \sigma \d W_t, && \d S_t = \sigma_t S_t \d W_t
    \end{align*}
    where \( W \) is a \( (\PP,\bF) \)-Brownian motino and the process \( \sigma = (\sigma_t)_{0\leq t\leq T} \) evolves independently of \( W \).
\begin{enumerate}[nolistsep,label=(\alph*)]
    \item Show that,
        \begin{align*}
            \EE[G(X_T-X_t)|\cF_t] = \EE[G(X_t-X_T)|\cF_t]
        \end{align*}
    \item Show that,
        \begin{align*}
            \EE[G(S_T)|\cF_t] = \EE[(S_T/S_t) G(S_t^2/S_T) | \cF_t]
        \end{align*}
        
\end{enumerate}

\end{problem}

\begin{solution}[Solution]
\begin{enumerate}[label=(\alph*)]
    \item 
    \item 
\end{enumerate}
\end{solution}


\begin{problem}[Practice Exam 8, Problem 1]
A bakery uses a two-step process to make chocolate cakes. The first step involves baking the cake and the second step involves frosting the cake. Baking takes an exponentially distributed amount of time with rate \( \mu_1 \) . After a cake is baked, it goes to the frosting machine. Frosting takes an exponentially distributed amount of time with rate \( \mu_2 \). The processing times at the oven and the frosting machine are independent random variables. Potential cakes arrive according to a Poisson process at rate \( \lambda \), however, a cake goes to the baking oven only if both the oven and the frosting machine are idle. If any of the two is busy, the cake simply exits the system. We wish to model this system as a continuous-time Markov chain.
\begin{enumerate}[nolistsep,label=(\alph*)]
    \item Precisely define the states of your continuous-time Markov chain (Hint: your model should have three states and note that there will never be two cakes in this system).
    \item Draw a transition diagram for your continuous-time Markov chain.
    \item Derive stationary distribution for your continuous-time Markov chain.
    \item Find the expected number of cakes in this system in steady-state.
\end{enumerate}
\end{problem}

\begin{solution}[Solution]
\begin{enumerate}[label=(\alph*)]
    \item 
    \item 
\end{enumerate}
\end{solution}


\begin{problem}[Practice Exam 8, Problem 2]
    Fix a probability space \( (\Omega,\cF,\PP) \) and a filtration \( \bF = (\cF_t)_{0\leq t\leq T} \). Define a procecss \( X \) and stoppting time \( \tau \) as follows,
    \begin{align*}
        \d X_t = \mu \d t + \sigma \d W_t, && \tau = \inf\{t\geq 0: X_t\notin(a,b)\}
    \end{align*}
    where \( W \) is a \( (\PP,\bF) \)-Brownian motion. Define the following Laplace Transform,
    \begin{align*}
        L(x;\lambda) := \EE[e^{-\lambda \tau} | X_0 = 0), && x\in(a,b)
    \end{align*}
    Derive the PDE satisfied by \( L \) and use this to find \( L \) explicitly.
\end{problem}

\begin{solution}[Solution]
\begin{enumerate}[label=(\alph*)]
    \item 
    \item 
\end{enumerate}
\end{solution}





\pagebreak
\section{Homework Problems}
\pagebreak

\begin{problem}[Exercise 4.1]
A six-sided die is rolled repeatedly. Which of the following a Markov chains? For those that are, find the one-step transition matrix. 
\begin{enumerate}[nolistsep,label=(\alph*)]
	\item \( X_n \) is the largest number rolled up to the nth roll. 
	\item \( X_n \) is the number of sixes rolled in the first \( n \) rolls. 
	\item At time \( n \), \( X_n \) is the time since the last six was rolled.
	\item At time \( n \), \( X_n \) is the time until the next six is rolled.
\end{enumerate}
\end{problem}

\begin{solution}[Solution]
\begin{enumerate}[label=(\alph*)]
	\item Yes.
        \begin{align*}
            P = 
            \left[\begin{array}{cccccc}
                1/6 & 1/6 & 1/6 & 1/6 & 1/6 & 1/6 \\
                & 2/6 & 1/6 & 1/6 & 1/6 & 1/6 \\
                && 36 & 1/6 & 1/6 & 1/6 \\
                &&& 4/6 & 1/6 & 1/6 \\
                &&&& 5/6 & 1/6 \\
                &&&& & 1
            \end{array}\right]
        \end{align*}
        
	\item Yes.
        \begin{align*}
            P=
            \left[\begin{array}{cccc}
                5/6 & 1/6 \\
                & 5/6 & 1/6 \\
                && \ddots & \ddots
            \end{array}\right]
        \end{align*}
        
    \item Yes. Suppose \( X_n = i \). The next roll is either a 6, in which case \( X_{n+1} = 0 \). Otherwise \( X_{n+1} = i+1 \).  
        \begin{align*}
            P = \left[\begin{array}{ccccc}
            1/6 & 5/6 \\
            1/6 & & 5/6 \\
            1/6 &  & & 5/6 \\
            \vdots & & & & \ddots
            \end{array}\right]
        \end{align*}
        
    \item Yes. Suppose \( X_n = 0 \). The probability of \( X_{n+1} = j \) is \( (1/6)(5/6)^j \) as you must not roll a 6 for \( j \) turns, and then must roll a \( 6 \) on the \( j \)-th. Suppose \( X_n = i > 0 \). Then the next step you will be on turn closer to rolling a 6. That is, \( X_{n+1} = i-1 \).
        \begin{align*}
            P =
            \left[\begin{array}{cccccc}
                \frac{1}{6} & \frac{1}{6}\left(\frac{5}{6}\right) & \frac{1}{6} \left( \frac{5}{6}^2 \right) & \frac{1}{6} \left( \frac{5}{6} \right)^3 & \cdots \\
                1 \\
                & 1 \\
                & & 1 \\
                & & & 1 \\
                & & & & \ddots
            \end{array}\right]
        \end{align*}
        
\end{enumerate}
\end{solution}

\begin{problem}[Exercise 4.2]
    Let \(Y_n=X_{2n} \). Compute the transition matrix for \( Y \) when 
    \begin{enumerate}[nolistsep,label=(\alph*)]
        \item \( X \) is a simple random walk (i.e., \( X \) increases by one with probability \( p \) and decreases by 1 with probability \( q \))
        \item \( X \) is a branching process where \( G \) is the generating function of the number of offspring from each individual
    \end{enumerate}
\end{problem}

\begin{solution}[Solution]
\begin{enumerate}[label=(\alph*)]
    \item
        In each step we can go down with probability \( q \) and then down again with probability \( q \) or up with probability \( p \). Alternatively we can go up with probability \( p \) and then down with probability \( q \) or up again with probability \( p \). 

        Therefore we will end up two spaces down with probability \( q^2 \), in the same position with probability \( qp+pq = 2pq \), or up two spaces with probability \( p^2 \). Thus,
        \begin{align*}
            p(i,j) = \begin{cases} p^2 & j=i+2 \\ 2pq & i=j \\ q^2 & j=i-2 \\ 0 & \text{otherwise}\end{cases} 
        \end{align*}
    \item
    We can obtain the exponents of a generating function \( G(s) = a_0 + a_1s+a_2s^2 + ... \) by,
        \begin{align*}
            a_n = \dfrac{1}{n!} \dfrac{d^n}{ds^n}\Big[G(s) \Big]_{s=0}
        \end{align*}

        The coefficient of the \( s^k \) term is the value of the probability mass function of \( X \) evaluated at \( k \).

        The generating function of \( Y \) is \( G(G(s)) = G_2(s) \) from the notes.

        For a branching process with current population \( k \), the population of the next generation will be \( X_1+X_2+...+X_k\), where each \( X_i \) is iid with distribution \( X \). Therefore,    
        \begin{align*}
            p(i,j) = \dfrac{1}{j!} \dfrac{d^n}{ds^n}\Big[ G_2(s)^i \Big]_{s=0}
        \end{align*}
\end{enumerate}
\end{solution}

\begin{problem}[Exercise 4.3]
Let \( X \) be a Markov chain with state space \( S \) and absorbing state \( k \) (i.e., \( p(k, j) = 0 \) for all \( j \in S \)). Suppose \( j\rightarrow k\) for all \( j\in S \). Show that all states other than \( k \) are transient.
\end{problem}

\begin{solution}[Solution]
Fix a state \( j\in S \). By definition of \( j\rightarrow k \), \( \exists N\geq 0 : p_N(j,k) \geq 0\). Since \( \{ X_N = k | X_0 = j \} \subseteq \{\forall n, X_n \neq j | X_0 = j \} \) we have,
\begin{align*}
    0 < p_N(j,k) = \PP(X_N = k | X_0 = j) \leq \PP(\forall n, X_n\neq j | X_0 = j)
\end{align*}

Therefore,
\begin{align*}
    \PP(\exists n\geq 0: X_n = j | X_0 = j) = 1 - \PP(\forall n, X_n \neq j|X_0 = j) < 1
\end{align*}

This proves state \( j \) istransient. \qed


\end{solution}

\begin{problem}[Exercise 4.4]
Suppose two distinct states \( i,j \) satisfy
\begin{align*}
    \PP(\tau_j<\tau_i | X_0 = i ) = \PP(\tau_i < \tau_j | X_0 = j)
\end{align*}
    where \( \tau_j = \inf\{ n\geq 1 : X_n = j \} \). Show that, if \( X_0=i \), the expected value of visits to \( j \) prior to returning to \( i \) is one.
\end{problem}

\begin{solution}[Solution]
Write
\begin{align*}
    p=\PP(\tau_j<\tau_i | X_0 = i ) = \PP(\tau_i < \tau_j | X_0 = j)
\end{align*}

That is,
\( p \) is the probability that we go to state \( j \) before state \( i \) give we are in state \( i \), and \( p \) is also the probability that we go to state \( i \) before state \( j \) given we are in state \( j \).

Then \( 1-p \) is the probability that we do not go to state \( i \) before returning state \( j \),0 given we start in state \( j \).

So \( (1-p)^k \) is the probability that we return to state \( j \) exactly \( k \) times before moving to state \( i \), given we start in state \( j \).

Let \( N \) be the number of visits to \( j \) prior to returning to \( i \) given we start in state \( i \). 

The probability that \( N = k\in\ZZ_{\geq 0} \) is the probability that starting from state \( i \) we go to state \( j \), return to state \( j \) \( (k-1) \) times without returning to state \( i \), and then return to state \( i \) without going to returning to state \( j \).

So \( \PP(N=k | X_0 = i) = p(1-p)^{k-1}p\). This is the probability mass function for \( N \) so,
\begin{align*}
    \EE[N] = \sum_{n=0}^{\infty} np^2(1-p)^{k-1} = p \sum_{n=0}^{\infty}n(1-p)^n = p \dfrac{p}{(1-(1-p))^2} = 1
\end{align*}
\end{solution}

\begin{problem}[Exercise 4.5]
Let X be a Markov chain with transition matrix,
\begin{align*}
    P = \left[\begin{array}{ccc}1-2p & 2p & 0 \\ p & 1-2p & p \\ 0 & 2p & 1-2p\end{array}\right], && p\in(0,1)
\end{align*}
    Find \( P^n \), the invariant distribution \( \pi \), and the mean-recurrence times \( \overline{\tau}_j \) for \( j=1,2,3 \).
\end{problem}

\begin{solution}[Solution]
    Note that \( P \) has eigendecomposition \( P = V\Lambda V^{-1} \) where,
    \begin{align*}
        \Lambda =
        \left[\begin{array}{rrr}1 \\ &1-4p \\ && 1-2p\end{array}\right]
        ,&&
        V = \left[\begin{array}{rrr} 1 & 1 & -1 \\ 1 & -1 & 0 \\ 1 & 1 & 1\end{array}\right]
    \end{align*}
    
    Therefore, \( P^n = V\Lambda^n V^{-1} \). Explicitly,
    \begin{align*}
        P^n = 
        \left[\begin{array}{rrr} 1 & 1 & -1 \\ 1 & -1 & 0 \\ 1 & 1 & 1\end{array}\right]
        \left[\begin{array}{rrr}1 \\ &1-4p \\ && 1-2p\end{array}\right]
        \left[\begin{array}{rrr}1/4 & 1/2 & 1/4 \\ 1/4 & -1/2 & 1/4 \\ -1/2 & 0 & 1/2\end{array}\right]
    \end{align*}
    
    Invariant distributions are linear combinations of left eigenvectors corresponding to eigenvalues of 1. In this case that is the first row of \( V^{-1} \). That is,
    \begin{align*}
        \pi = \left[\begin{array}{rrr}\frac{1}{4} & \frac{1}{2} & \\frac{1}{4}\end{array}\right]
    \end{align*}

    Finally, since the invariant distribution is unique, by Theorem we have,
    \begin{align*}
        \overline{\tau}_i = \frac{1}{\pi(i)}
    \end{align*}
    
    
\end{solution}


\begin{problem}[Exercise 4.6]
    Let \( X_n \) be the number of mistakes in the \( n \)-th addition of a book. Between the \( n \)-th and the \( (n+1) \)-th addition an editor corrects each mistake independently with probability \( p \) and introduces \( Y_n \) new mistakes where the\( (Y_n) \) are iid and Poisson distributed with parameter \( \lambda \). Find the invariant distribution \( \pi \) of the number of mistakes in the book. 
\end{problem}

\begin{solution}[Solution]
Let \( M_{n,k} \) be distributed as \( \operatorname{Ber}(1-p) \) so that \( M_k \) is 0 if this mistake is corrected, and 1 otherwise. Let \( Y_n \) be Poisson distributed with parameter \( \lambda \). Then,
\begin{align*}
    X_{n+1} = Y_n + \sum_{k-1}^{X_n} M_k 
\end{align*}

Each \( M_{n,k} \) has generating function,
 \begin{align*}
     G_{M_{n,k}} = p+(1-p)s = 1-q+qs = 1-q(1-s) 
 \end{align*}

Similarly. \( Y_n \) has generating function,
 \begin{align*} 
     G_Y(s) = \sum_{k=0}^{\infty} e^{-\lambda} \lambda^k/k! s^k = e^{-\lambda}e^{s\lambda} = e^{\lambda(s-1)}
 \end{align*} 
 

Therefore \( X_{n+1} \) has generating function,
\begin{align*}
    G_{n+1}(s) &= G_Y(s) \EE\left[ s^{M_{k,1}+M_{k,2}+...+M_{k,X_n}} \right] \\
    &= G_Y(s) \EE\left[\EE\left[ s^{M_{k,1}+M_{k,2}+...+M_{k,X_n}}\right] | X_n\right] \\
    &= G_Y(s) \EE \left[ (1-q(1-s))^{X_n} \right] \\
    &= G_Y(s) G_n(1-q(1-s)) \\
\end{align*}

First observe \( 1-q^i(1-(1-q(1-s))) = 1-q^{i+1}(1-s) \). We now use the relation \( G_{n+1}(s) = G_Y(s)G_n(1-q(1-s)) \) and the fact that \( G_0(s) = 1 \) to calculate,
\begin{align*}
    G_{n+1}(s) &= G_Y(s) G_n(1-q(1-s)) \\
%    &= G_Y(s) G_Y(1-q(1-s))G_{n-1}(1-q(1-(1-q(1-s)))) \\
    &= G_Y(s) G_Y(1-q(1-s))G_{n-1}(1-q^2(1-s)) \\
    &= G_Y(s) G_Y(1-q(1-s)) G_Y(1-q^2(1-s)) G_{n-2}(1-q^3(1-s)) \\
    &\mathrel{\makebox[\widthof{=}]{\vdots}} \\
    &= \prod_{i=0}^{n}G_Y(1-q^i(1-s)) 
\end{align*}

Then,
\begin{align*}
    \lim_{n\to\infty} G_{n}(s) &= \lim_{n\to\infty} G_{n+1}(s) \\
    &= \lim_{n\to\infty} \prod_{i=0}^{n}G_Y(1-q^i(1-s)) \\
    &= \lim_{n\to\infty} \prod_{i=0}^{n} \exp\left( \lambda(-q^i(1-s)) \right)\\
    &= \exp \left( \sum_{i=0}^{\infty} \lambda(-q^i(1-s) \right) \\
    &= \exp \left( \lambda(s-1) \dfrac{1}{1-q}\right) \\
    &= \exp \left( \dfrac{\lambda}{p}(s-1) \right) \\
\end{align*}

Thus, \( G_n(S) \) converges to the generating function of a Poisson random variable with parameter \( \lambda/p \).

Then \( X_n \) converges to a random variable distributed like a Poisson random variable with parameter \( \lambda/p \). The random variable for which \( X_n \) converges to must be the variable corresponding to the stationary distribution. Therefore, the stationary distribution is distributed like the probability mass function of this random variable. That is,
\begin{align*}
    \pi(k) = e^{-\lambda/p} \dfrac{(\lambda/p)^k}{k!}
\end{align*}

In the limit \( p\to 1 \), where we correct all mistakes, the stationary distribution looks like a Poisson distribution with parameter \( \lambda \). In the limit \( \lambda \to 0 \) so we do not make any new mistakes, \( \pi(0)\to 1 \) as expected.
\end{solution}


\begin{problem}[Exercise 4.7]
Give an example of a transition matrix \( P \) that admits multiple stationary distributions \( \pi \).
\end{problem}

\begin{solution}[Solution]
Define \( P \) to be the identity matrix. Then any distribution is a stationary distribution.
\end{solution}


\begin{problem}[Exercise 4.8]
    A Markov chain on \( S=\{0,1,2,...,n\} \) has transition probabilities \( p(0,0) = 1-\lambda_0 \), \( p(i,i+1) = \lambda_i \) and \( p(i+1,i) = \mu_{i+1} \) for \( i=0,1,...,n-1 \), and \( p(n,n)=1-\mu_n \). Show that the process is reversible in equilibrium.
\end{problem}

\begin{solution}[Solution]
We assume all entries not specified are zero. (I heard this is the intent, however I wonder why we are given \( \mu_j\) when \( \mu_j=1-\lambda_j \) for all \( j \)). We write the matrix \( P \) as,

Write \( \mu_n=1-\lambda_n  \). Thus, \( \mu_i=1-\lambda_i \) for \( i=1, ..., n \) as the sum of each row must be 1 (making the assumption that all entries not specified at zero). 
    {\tiny
    \begin{align*}
    P = \left[\begin{array}{rrrrrr}
        1-\lambda_0 & \lambda_0 & & &\\
        \mu_1 & & \lambda_1 & & \\
        & \mu_2 & & \lambda_2 & \\
        & & \mu_3 \\\\
        & & & & & \lambda_{n-1} \\
        & & & & \mu_n & 1-\mu_n
    \end{array}\right]
      = \left[\begin{array}{rrrrrr}
        1-\lambda_0 & \lambda_0 & & &\\
        1-\lambda_1 & & \lambda_1 & & \\
        & 1-\lambda_2 & & \lambda_2 & \\
        && 1-\lambda_3\\\\
        & & & & & \lambda_{n-1} \\
        & & & & 1-\lambda_n & \lambda_n
    \end{array}\right]
\end{align*}
    }

This chain is irreducible and finite so a unique invariant distribution \( \pi \) exists. Write \( \pi=[\pi_0,\pi_1, ..., \pi_n] \). Then \( \pi P = \pi \). That is,
\begin{align*}
    \pi P = \left[\begin{array}{r}
        \pi_0(1-\lambda_0)+\pi_1(1-\lambda_1) \\
        \pi_0 \lambda_0 + \pi_2(1-\lambda_2) \\ 
        \pi_1\lambda_1+\pi_3(1-\lambda_3) \\
        \vdots \\ 
        \vdots \\
        \pi_{n-1}\lambda_{n-1}+\pi_n\lambda_n
    \end{array}\right]^T
= \left[\begin{array}{r}
    \pi_0 \\ 
    \pi_1 \\
    \pi_2 \\
    \vdots \\
    \pi_j \\
    \vdots \\
    \pi_n
\end{array}\right]^T
\end{align*}

\begin{align*}
    \pi_1 &= \lambda_0\pi_0 / (1-\lambda_1) 
        & \lambda_0\pi_0 &= \pi_1(1-\lambda_1) \\
    \pi_2 &= (\pi_1-\pi_0\lambda_0)/(1-\lambda_2) = \pi_1\lambda_1/(1-\lambda_2) 
        & \lambda_1\pi_1 &= \pi_2(1-\lambda_2) \\
    \pi_3 &= (\pi_2-\pi_1\lambda_1)/(1-\lambda_3) = \pi_2\lambda_2/(1-\lambda_3) 
        & \lambda_2\pi_2 &= \pi_3(1-\lambda_3) \\
    \vdots \\
    \pi_{j+1} &= (\pi_j-\pi_{j-1}\lambda_{j-1})/(1-\lambda_{j+1}) = \pi_j\lambda_j/(1-\lambda_{j+1}) & \lambda_j\pi_j &= \pi_{j+1}(1-\lambda_{j+1})\\
    \vdots \\
    \pi_n &= (\pi_{n-1}\lambda_{n-1})/(1-\lambda_n) 
        & \pi_{n-1}\lambda_{n-1} &= \pi_n(1-\lambda_n)
\end{align*}

Observing the equations on the right hand side we have that for \( i=1,2,...,n-1 \),
\begin{align*}
    \pi_{i}p(i,i+p) = \pi_{i+1}p(i+1,i)
\end{align*}

We now show the detail balance condition. In particular, we must show, 
\begin{align*}
    \pi_ip(i,j) = \pi_jp(j,i)  &&\text{ for all } i,j 
\end{align*}
However, for \( j\notin\{i-1,i+1\} \) we have \( p(i,j)=0 \). Therefore, for this matrix the previous condition is equivalent to 
\begin{align*}
    \pi_i p(i,i+1) = \pi_{i+1}p(i+1,i) &&\text{ for } i=1,2,..., n-1
\end{align*}

We have shown that these equations hold for all \( i=1,2,...,n-1 \). 

This proves \( \pi \) is in detailed balance with \( P \), and so this process is reversible in equilibrium. \qed

\end{solution}


\pagebreak
\begin{problem}[Exercise 5.1]
Patients arrive at an emergency room as a Poisson process with intensity \( \lambda \). The time to treat each patient is an independent exponential random variable with parameter \( \mu \). Let \( X= (X_t)_{t\geq 0} \) be the number of patients in the system (either being treated or waiting). Write down the generator of \( X \). Show that \( X \) has an invariant distribution \( \pi \) if and only if \( \lambda<\mu \). Find \( \pi \). What is the total expected time (waiting + treatment) a patient waits when the system is in its invariant distribution?
\end{problem}

\begin{solution}[Solution]
In some small time interval \( s \) there is probability \( \lambda s + \mathcal{O}(s^2) \) that a patient arrives, probability \( 1-\lambda s + \mathcal{O}^2 \) that a patient does not arrive, and probability \( \mathcal{O}(s^2) \) that multiple patients arrive.

If there are patients, in this times there is also probability \( \mu s + \mathcal{O}(s^2) \) that a patient is treated, probability \( 1- \mu s + \mathcal{O}(s^2) \) that a patient is not treated, and probability \( \mathcal{O}(s^2) \) that more than one (if possible) patients are treated.

Note that any moves which have more than one transition such as a patient arriving, and a patient being treated are all \( \mathcal{O}(s^2) \).

Suppose there are no patients at time \( t \). The probability of transitioning to \( j \) patients after a short time \( s \) is given by,
\begin{align*}
    \PP(X_{t+s} = j | X_{t} = 0) = 
    \begin{cases}
        \lambda s + \mathcal{O}(s^2) & j=1 \\
        1-\lambda s + \mathcal{O}(s^2) & j=0 \\
        \mathcal{O}(s^2) & \text{otherwise}
    \end{cases}
\end{align*}

Now suppose there are \( i>0 \) patients at time \( t \). The probability of transitioning to \( j \) patients after a short time \( s \) is given by,
\begin{align*}
    \PP(X_{t+s} = j | X_{t} = i) = 
    \begin{cases}
        (\lambda s + \mathcal{O}(s^2))(1-\mu s + \mathcal{O}(s^2)) & j=i+1 \\
        (1-\lambda s + \mathcal{O}(s^2))(1-\mu s + \mathcal{O}(s^2)) + \mathcal{O}(s^2) & j=i \\
        (1-\lambda s + \mathcal{O}(s^2))(\mu s + \mathcal{O}(s^2)) & j=i-1 \\
        \mathcal{O}(s^2) & \text{otherwise}
    \end{cases}
\end{align*}

This is simplified as,
\begin{align*}
    \PP(X_{t+s}=j | X_t = i) = 
    \begin{cases}
        \lambda s + \mathcal{O}(s^2) & j=i+1 \\
        1-\lambda s -\mu s +\mathcal{O}(s^2) & j=i \\
        \mu s + \mathcal{O}(s^2) & j=i-1 \\
        \mathcal{O}(s^2) & \text{otherwise}
    \end{cases}
\end{align*}


This gives,
\begin{align*}
    G = 
    \left[\begin{array}{cccccc}
        -\lambda & \lambda \\
        \mu & -(\lambda+\mu) & \lambda \\
        &  \mu & -(\lambda+\mu) & \lambda \\
        & &  \mu & -(\lambda+\mu) & \lambda & \cdots\\
        & & & \vdots & \vdots & \ddots 
    \end{array}\right]
\end{align*}

We recognize this as a birth-death process (a bit ironic in the context of an emergency room) with \( \lambda_i = \lambda \) and \( \mu_i=\mu \).

Then if a stationary distribution \( \pi \) exists, for \( n\in\ZZ_{>0} \),
\begin{align*}
    \pi(n>0) = \left(\dfrac{\lambda}{\mu}\right)^{n} \pi(0)
\end{align*}
and
\begin{align*}
    \pi(0) = \left( 1+\sum_{n=1}^{\infty} \left( \dfrac{\lambda}{\mu} \right)^n \right)^{-1} = \left( \sum_{n=0}^{\infty} \left( \dfrac{\lambda}{\mu} \right)^n \right)^{-1}
\end{align*}

This is a geometric series which is convergent exactly when \( \lambda/\mu < 1 \). That is, when \( \lambda < \mu \). In this case,
\begin{align*}
    \pi(0) = \left( \sum_{n=0}^{\infty} \left( \dfrac{\lambda}{\mu} \right)^n \right)^{-1} = \left(\dfrac{\mu}{\mu-\lambda}\right)^{-1} = \dfrac{\mu-\lambda}{\mu}
\end{align*}


We condition on knowing the number of people on the queue. Suppose there are \( n \) people in the queue when a patient arrives. Then the patient will have to wait a random time distributed as the sum of \( n \) exponential random variables with parameter \( \mu \) to be treated and one more to finish treatment. The expectation of each of each exponential random variable is \( 1/\mu \), so the patient waits an expected time of \( (n+1)/\mu \).

In equilibrium, the probability that there are \( n \) people in the queue when a patient arrives is \( \pi(n) \). 

Therefore, the expected wait time is,
\begin{align*}
    \sum_{n=0}^{\infty} \pi(n) \dfrac{(n+1)}{\mu} 
    %= \sum_{n=0}^{\infty} \left( \dfrac{\lambda}{\mu} \right)^n \left(\dfrac{\mu-\lambda}{\mu}\right)\left(\dfrac{n+1}{\mu} \right) 
    = \dfrac{\mu-\lambda}{\mu^2}\sum_{n=0}^{\infty} \left( \dfrac{\lambda}{\mu} \right)^n (n+1) 
    = \dfrac{\mu-\lambda}{\mu^2} \left( \dfrac{\mu\lambda}{(\mu-\lambda)^2}+\dfrac{\mu}{\mu-\lambda} \right)
    = \dfrac{1}{\mu-\lambda}
\end{align*}
\end{solution}

\pagebreak
\begin{problem}[Exercise 5.2]
    Let \( X = (X_t)_{t\geq 0} \) be a Markov chain with stationary distribution \( \pi \). Let \( N \) be an independent Poisson process with intensity \( \lambda \) and denote by \( \tau_n \) the time of the \( n \)-th arrival of \( N \). Define \( Y_n:=X_{\tau_n+} \) (i.e., \( Y_n \) is the value of \( X \) immediately after the \( n \)-th jump). Show that \( Y \) is a discrete time Markov chain with the same stationary distribution as \( X \).
\end{problem}

\begin{solution}
It is obvious that \( Y \) is Markov, as given the present, the future is independent of the past. We add a bit more rigor below.

Fix a probability space \( (\Omega, \mathcal{F}, \PP) \).
By hypothesis \( X_t \) is a Markov process. That is, for a filtration \( (\mathcal{F}_s)_{s\in[0,T]} \), for \( 0\leq s\leq t\leq T \), and for every non-negative Borel measurable function \( f \), 
\begin{align*}
    \EE[f(X_t) | \mathcal{F}_s] = \EE[f(X_t)|X_s]
\end{align*}

Let \( \mathcal{F}'_n = \mathcal{F}_{\tau_n+} \) be a sub-\( \sigma \)-algebra of \( \mathcal{F} \). Then clearly \( (\mathcal{F}'_n) \) is a filtration. Let \( f \) be any non-negative Borel measurable function. Then,
\begin{align*}
    \EE[f(Y_n) | \mathcal{F}'_m] = \EE[f(X_{\tau_n+}) | \mathcal{F}_{\tau_m+}] = \EE[f(X_{\tau_n+}) | X_{\tau_m+}] = \EE[f(Y_n) | Y_m]
\end{align*}

This means \( Y \) is Markov, and clearly \( Y \) is discrete time. Therefore \( Y \) is a discrete time Markov chain.

%A discrete time Markov chain is a Markov process process with countable state space \( S \).


Note we assume \( X \) is time homogeneous.

Suppose \( X \) has stationary distribution \( \pi \). Then for all \( 0\leq t \leq T \), \( \pi P_t = \pi \), where,
\begin{align*}
    (P_t)_{i,j} = \PP(X_t = j | X_0 = i)
\end{align*}

Thus, the one step probability transition matrix, denoted \( \tilde{P} \), for \( Y \) is,
\begin{align*}
    \tilde{P}_{i,j} = \PP(Y_1=j | Y_0 = i) = \PP( X_{\tau_1+}=j | X_0 = i) = (P_{\tau_1})_{i,j}
\end{align*}

This means \( \pi \tilde{P} = \pi \), so \( \pi \) is a stationary distribution of \( Y \).
\end{solution}

\pagebreak
\begin{problem}[Exercise 5.3]
    Let \( X=(X_t)_{t\geq 0} \) be a Markov chain with state space \( S=\{0,1,2,...\} \) and generator \( G \) whose \( i \)-th row has entries
    \begin{align*}
        g_{i,i-1} = i\mu && g_{i,i} = -i\mu-\lambda && g_{i,i+1} = \lambda,
    \end{align*}
    with all other entries being zero (the zeroth row has only two entries: \( g_{0,0} \) and \( g_{0,1} \)). Assume \( X_0=j \). Find \( G_{X_t}(s) := \EE s^{X_t} \). What is the distribution of \( X_t \) as \( t\to\infty \)?
\end{problem}

\begin{solution}[Solution]
We have \( G \) in matrix form,
\begin{align*}
    G = 
    \left[\begin{array}{cccccc}
        -\lambda & \lambda \\
        \mu & -(\mu+\lambda) & \lambda \\
        & 2\mu & -(2\mu+\lambda) & \lambda \\
        & & 3\mu & -3(\mu+\lambda) & \lambda & \cdots \\
        & & & \vdots & \vdots & \ddots
    \end{array}\right]
\end{align*}

We wish to find the transition semi group \( P_t \). We know this can be derived from the Kolmogorov forward equations. That is,
\begin{align*}
    \dfrac{d}{dt}P_t = P_t G
\end{align*}

With the assumption that \( X_0 = i \) ({\em I am using \( i \) rather than \( j \) like the problem statement since this is the standard way of doing things}) we have,
\begin{align*}
    \dfrac{d}{dt}p_t(i,0) &= \sum_{k=0}^{\infty}p_t(i,k)g(k,0) %= p(i,0)g(0,0) + p(i,1)g(1,0) 
    = -\lambda p_t(i,0) + \mu p_t(i,1) \\
    \dfrac{d}{dt}p_t(i,j) &= \sum_{k=0}^{\infty}p_t(i,k)g_t(k,j) 
    = \lambda p_t(i,j-1) -(j\mu+\lambda) p_t(i,j) + (j+1)\mu p_t(i,j+1) \tag*{\( j\geq 1 \)}
\end{align*}

We multiply the \( j \)-th equation by \( s^j \).
This gives,
\begin{align*}
    \sum_{j=0}^{\infty} \dfrac{\partial}{\partial t} p_t(i,j)s^j
    = \sum_{j=1}^{\infty} \left[ \lambda p_t(i,j-1)s^j\right] - \sum_{j=0}^{\infty}\left[ (j\mu-\lambda)p_t(i,j)s^j\right] + \sum_{j=0}^{\infty} \left[ (j+1)\mu p_t(i,j+1)s^j \right]
\end{align*}

Summing the left hand sides gives,
\begin{align*}
    \sum_{j=0}^{\infty} \dfrac{\partial}{\partial t} p_t(i,j)s^j
    = \dfrac{\partial}{\partial t} \sum_{j=0}^{\infty} p_t(i,j)s^j
    = \dfrac{\partial}{\partial t} G_{X_t}(s)
\end{align*}

The first term of the right hand side gives,
\begin{align*}
    \sum_{j=1}^{\infty}\lambda p_t(i,j-1)s^j 
    &= \lambda s \sum_{j=1}^{\infty} p_t(i,j-1)s^{j-1}
    = \lambda s \sum_{j=0}^{\infty} p_t(i,j)s^j
    = \lambda s G_{X_t}(s)
\end{align*}

The negative of the first part of the second term of the right hand side gives,
\begin{align*}
    \sum_{j=0}^{\infty} j\mu p_t(i,j)s^j
    = s\mu  \sum_{j=0}^{\infty} j p_t(i,j)s^{j-1}
    = s\mu \sum_{j=0}^{\infty} \dfrac{\partial}{\partial s} p_t(i,j) s^{j}
    = s\mu \dfrac{\partial}{\partial s} \sum_{j=0}^{\infty} p_t(i,j) s^{j}
    = s\mu \dfrac{\partial}{\partial s} G_{X_t}(s)
\end{align*}

The negative of the second part of the second term of the right hand side gives,
\begin{align*}
    \sum_{j=0}^{\infty} \lambda p_t(i,j)s^{j}
    = \lambda \sum_{j=0}^{\infty} p_t(i,j) s^{j}
    = \lambda G_{X_t}(s)
\end{align*}

The third term of the right hand side gives,
\begin{align*}
    \sum_{j=1}^{\infty} (j+1)\mu p_t(i,j+1) s^{j}
%    &= \mu \sum_{j=1}^{\infty} (j+1) p_t(i,j+1)s^{j}
    = \mu \sum_{j=1}^{\infty} \dfrac{\partial}{\partial s} p_t(i,j+1) s^{j+1} 
    = \mu \dfrac{\partial}{\partial s} \sum_{j=0}^{\infty} p_t(i,j)s^j
    = \mu \dfrac{\partial}{\partial s}G_{X_t}(s)
\end{align*}

Putting these results together we have,
\begin{align*}
    \dfrac{\partial}{\partial t} G_{X_t}(s)
    = \left[ \lambda s - s\mu \dfrac{\partial}{\partial s} - \lambda + \mu \dfrac{\partial}{\partial s} \right] G_{X_t}(s) 
\end{align*}

Since \( X_0 = j \) we have initial condition, 
\begin{align*}
    G_{X_0}(s) = s^j
\end{align*}

We solve with Mathematica by,
\begin{lstlisting}
DSolve[{
    D[G[s,t],t]==\[Lambda] s G[s,t]-s \[Mu] D[G[s,t],s]-\[Lambda] G[s,t]+\[Mu] D[G[s,t],s],
    G[s,0]==s^j
    },G[s,t],{s,t}]//FullSimplify
\end{lstlisting}

This yields,
\begin{align*}
    G_{X_t}(s) = \left((s-1) e^{-\mu t}+1\right)^j \exp \left[ \frac{\lambda  (s-1) e^{\mu  (-t)} \left(e^{\mu  t}-1\right)}{\mu } \right]
\end{align*}

We find the limit as \( t\to\infty \) with Mathematica by,
\begin{lstlisting}
Limit[E^((E^(-t \[Mu]) (-1+E^(t \[Mu])) (-1+s) \[Lambda])/\[Mu]) (1+E^(-t \[Mu]) (-1+s))^j,{t->\[Infinity]},Assumptions->{\[Lambda]>0,\[Mu]>0}]
\end{lstlisting}

This yields,
\begin{align*}
    G_{X_\infty}(s) = \lim_{t\to\infty} G_{X_t}(s) = e^{\frac{\lambda}{\mu}(s-1)} 
\end{align*}

So \( X_\infty = \lim_{t\to\infty} X_t \) is a Poission random variable with parameter \( \lambda/\mu \).
\end{solution}

\pagebreak
\begin{problem}[Exercise 5.4]
    Let \( N \) be a time-inhomogeneous Poisson process with intensity function \( \lambda(t) \). That is, the probability of a jump of size one in the time interval \( (t,t+ dt) \) is \( \lambda(t)dt \) and the probability of two jumps in that interval of time is \( \mathcal{O}(dt^2) \). Write down the Kolmogorov forward and backward equations of \( N \) and solve them. Let \( N_0 = 0 \) and let \( \tau_1 \) be the time of the first jump of \( N \). If \( \lambda(t) = c/(1 + t) \) show that \( \EE\tau_1< \infty \) if and only if \( c >1 \).
\end{problem}

\begin{solution}[Solution]
Based on the definition of the generator and the given transition probabilities we have,
\begin{align*}
    G(t) = 
    \left[\begin{array}{rrrrr}
        -\lambda(t) & \lambda(t) \\
        & -\lambda(t) & \lambda(t) \\
        & & -\lambda(t) & \lambda(t) & \cdots \\
        & & \vdots & \vdots & \ddots
    \end{array}\right]
\end{align*}

For \( t\geq s \) we define, 
\begin{align*}
    p_{s,t}(i,j) = \PP( N_t = j | N_s = i)
\end{align*}

We first derive the Kolmogorov forward equations. We consider,
\begin{align*}
    p_{s,t+\Delta t} &= \PP( N_{t+\Delta t} = j | N_s = i) \\
    &= \sum_{k}^{} \PP(N_{t+\Delta t}=j | N_t = k)\PP(N_{t}=k | N_s=i) \\
    &= \begin{cases}
        \lambda(t)\Delta t p_{s,t}(i,j-1) 
        +(1-\lambda(t)\Delta t) p_{s,t}(i,j) + \mathcal{O}(\Delta t^2) & j > i \\
        (1-\lambda(t)\Delta t) p_{s,t}(i,j) +\mathcal{O}(\Delta t^2) & j=i \\
        0 & j < i
    \end{cases}
\end{align*}

Therefore,
\begin{align*}
    \dfrac{p_{s,t+\Delta t}(i,j) - p_{s,t}(i,j)}{\Delta t} 
    &= \begin{cases}
        \lambda(t)\Delta t p_{s,t}(i,j-1) 
        -\lambda(t)\Delta t p_{s,t}(i,j) + \mathcal{O}(\Delta t^2) & j > i \\
        -\lambda(t)\Delta t p_{s,t}(i,j) +\mathcal{O}(\Delta t^2) & j=i \\
        0 & j < i
    \end{cases}
\end{align*}

Taking the limit as \( \Delta t \to 0  \) we have,
\begin{align*}
    \dfrac{\partial}{\partial t}p_{s,t}(i,j) = 
    \begin{cases}
        \lambda(t) p_{s,t}(i,j-1) 
        -\lambda(t)p_{s,t}(i,j) & j > i \\
        -\lambda(t)p_{s,t}(i,j) & j=i \\
        0 & j < i
    \end{cases}
\end{align*}

Fix \( i \). Noting that \( G_F(x) \) is also a function of \( s,t \) and \( j \), we have,
\begin{align*}
    G_F(x) = \sum_{j=0}^{\infty} \PP(N_t = j | N_s = i) x^{j} = \sum_{j=i}^{\infty}p_{s,t}(i,j) x^{j} 
\end{align*}

Thus, multiplying the \( j \)-th KFE by \( x^j \) and summing, we have,
\begin{align*}
    \dfrac{\partial}{\partial t} \sum_{j=i}^{\infty} p_{s,t}(i,j)x^{j}
    =\sum_{j=i}^{\infty} \dfrac{\partial}{\partial t} p_{s,t}(i,j)x^j 
    &= \sum_{j=i+1}^{\infty} \lambda(t) p_{s,t}(i,j-1) x^{j} + \sum_{j=i}^{\infty} (-\lambda(t)) p_{s,t}(i,j) x^j \\
    &= \lambda(t) x\sum_{j=i}^{\infty} p_{s,t}(i,j) x^{j} -\lambda(t) \sum_{j=i}^{\infty} p_{s,t}(i,j) x^j \\
\end{align*}

Therefore,
\begin{align*}
    \dfrac{\partial}{\partial t} G_{F}(x) 
    = \lambda(t) x G_{F}(x) - \lambda(t) G_{F}(x) 
    = \lambda(t) (x-1)G_{F}(x)
\end{align*}

We have initial condition \( N_s = i \), so \( G_B(x) = x^i \) when \( s=t \). 

We solve with Mathematica as,
\begin{lstlisting}
DSolve[{D[G[s, t], t] == \[Lambda][t] (x - 1) G[s, t],
   G[s, s] == x^i
   }, G[s, t], {s, t}] // FullSimplify
\end{lstlisting}

This gives,
\begin{align*}
    G_F(x) = x^i\exp \left( (x-1) \int_{s}^{t}\lambda(z)\d z \right)
\end{align*}

Write \( I = \int_{s}^{t}\lambda(z)dz \). Then,
\begin{align*}
    G_F(x) = e^{-I} x^i e^{Ix} = e^{-I} x^i \sum_{k=0}^{\infty} \dfrac{1}{k!}(Ix)^k 
    = e^{-I} \sum_{k=0}^{\infty}\dfrac{1}{k!}I^kx^{k+i} 
    = e^{-I} \sum_{j=i}^{\infty} \dfrac{I^{j-i}}{(j-i)!}x^{j}
\end{align*}

Therefore, from the definition of the Generating function we have,
\begin{align*}
    P_{s,t}(i,j) 
    = \PP(N_t=j | N_s = i) 
    =  \dfrac{1}{(j-i)!} \left[ \int_{s}^{t}\lambda(z)\d z \right]^{j-i}\exp \left( -\int_{s}^{t} \lambda(z)\d z \right)
\end{align*}



We now derive the Kolmogorov Backward equations. We consider,
\begin{align*}
    p_{s-\Delta s,t} &= \PP( N_{t} = j | N_{s-\Delta s} = i) \\
    &= \sum_{k}^{} \PP(N_{t}=j | N_{s}t = k)\PP(N_{s}=k | N_{s-\Delta s}=i) \\
    &= \begin{cases}
        \lambda(s)\Delta s p_{s,t}(i+1,j) 
        +(1-\lambda(s)\Delta s) p_{s,t}(i,j) + \mathcal{O}(\Delta s^2) & j > i \\
        (1-\lambda(s)\Delta s) p_{s,t}(i,j) +\mathcal{O}(\Delta s^2) & j=i \\
        0 & j < i
    \end{cases}
\end{align*}

Therefore,
\begin{align*}
    \dfrac{p_{s-\Delta s,t}(i,j) - p_{s,t}(i,j)}{\Delta s} 
    &= \begin{cases}
        \lambda(s)\Delta t p_{s,t}(i+1,j) 
        -\lambda(s)\Delta t p_{s,t}(i,j) + \mathcal{O}(\Delta s^2) & j > i \\
        -\lambda(s)\Delta t p_{s,t}(i,j) +\mathcal{O}(\Delta s^2) & j=i \\
        0 & j < i
    \end{cases}
\end{align*}

Taking the limit as \( \Delta s \to 0  \) we have,
\begin{align*}
    -\dfrac{\partial}{\partial s}p_{s,t}(i,j) = 
    \begin{cases}
        \lambda(s) p_{s,t}(i+1,j) 
        -\lambda(s)p_{s,t}(i,j) & j > i \\
        -\lambda(s)p_{s,t}(i,j) & j=i \\
        0 & j < i
    \end{cases}
\end{align*}

Fix \( i \). Noting that \( G_B(x) \) is also a function of \( s,t \) and \( j \), we have,
\begin{align*}
    G_B(x) = \sum_{j=0}^{\infty} \PP(N_t = j | N_s = i) x^{j} = \sum_{j=i}^{\infty}p_{s,t}(i,j) x^{j} 
\end{align*}

Thus, multiplying the \( j \)-th KBE by \( x^j \) and summing, we have,
\begin{align*}
    -\dfrac{\partial}{\partial s} \sum_{j=i}^{\infty} p_{s,t}(i,j)x^j = 
    -\sum_{j=i}^{\infty} \dfrac{\partial}{\partial s} p_{s,t}(i,j)x^j 
    &= \sum_{j=i+1}^{\infty} \lambda(s) p_{s,t}(i+1,j) x^{j} + \sum_{j=i}^{\infty} (-\lambda(s)) p_{s,t}(i,j) x^j \\
    &= \sum_{j=i+1}^{\infty} \lambda(s) p_{s,t}(i,j-1) x^{j} + \sum_{j=i}^{\infty} (-\lambda(s)) p_{s,t}(i,j) x^j \\
    &= \lambda(s) x\sum_{j=i}^{\infty} p_{s,t}(i,j) x^{j} -\lambda(s) \sum_{j=i}^{\infty} p_{s,t}(i,j) x^j \\
\end{align*}

Therefore,
\begin{align*}
    \dfrac{\partial}{\partial s} G_{B}(x) 
    = -\lambda(s) x G_{B}(x) + \lambda(s) G_{B}(x) 
    = -\lambda(s) (x-1)G_{B}(x)
\end{align*}

From the result for \( G_F(x) \) we know,
\begin{align*}
    G_B(x) = x^i\exp \left( -(x-1) \int_{t}^{s}\lambda(z)\d z \right) = x^i \exp \left( (x-1) \int_{s}^{t}\lambda(z)\d z \right) = G_F(x)
\end{align*}




We now show that for \( \lambda(t) = c/(1+t) \), that \( \EE \tau_1 < \infty \) if and only if \( c<1 \).
Indeed,
\begin{align*}
    \int_{0}^{t}\lambda(z)\d z = \int_{0}^{t} \dfrac{c}{1+z}\d z = c\ln(1+t) - c\ln(1) = c \ln(1+t)
\end{align*}

Therefore,
\begin{align*}
    \EE[\tau_1] 
    =  \int_{0}^{\infty} \PP(\tau_1 > t) \d t 
    = \int_{0}^{\infty} \PP(N_t=0 | N_0 = 0) \d t 
    = \int_{0}^{\infty} \exp(-c\ln(1+t))\d t 
    = \int_{0}^{\infty} \dfrac{\d t}{(1+t)^c}
\end{align*}

This is convergent if and only if \( c>1 \).
\end{solution}

\pagebreak
\begin{problem}[Exercise 5.5]
    Let \( N_t \) be a Poisson process with a random intensity \( \Lambda \) which is equal to \( \lambda_1 \) with probability \( p \) and \( \lambda_2 \) with probability \( 1-p \). Find \( G_{N_t}(s) = \EE s^{N_t} \). What is the mean and variance of \( N_t \)?
\end{problem}

\begin{solution}[Solution]
Recall the generating function for a Poisson process with intensity \( \lambda \) is,
\begin{align*}
    G(s) = e^{-\lambda t(1-s)}
\end{align*}

%\textbf{DO I NEED TO DERIVE THIS????}

Therefore,
\begin{align*}
    G_{N_t}(s) 
    = \EE \left[ s^{N_t} \right] 
    = \EE \left[ \EE \left[ s^{N_t} \right] \Big| \Lambda \right] 
    = \EE\left[ e^{-\Lambda t(1-s)} \Big| \Lambda \right]
    = pe^{-\lambda_1t(1-s)} + (1-p) e^{-\lambda_2(1-s)}
\end{align*}

We use Mathematica to caluculate moments,
\begin{lstlisting}
GNt[s_]:=p Exp[-\[Lambda]1 t (1-s)]+(1-p)Exp[-\[Lambda]2 t(1-s)]
D[GNt[s],{s,1}]/.{s->1}
D[GNt[s],{s,2}]-D[GNt[s],{s,1}]^2+D[GNt[s],{s,1}]/.{s->1}
\end{lstlisting}

This yields,
\begin{align*}
    \mu &= G'_{N_t}(1) = p \lambda_1 t  + (1-p) \lambda_2 t 
    \\\sigma^2 &= G''_{N_t}(1) - [G'_{N_t}(1)]^2 + G'_{N_t}(1) =  
    p (\lambda_1 t)^2 + (1-p)(\lambda_2 t)^2 - \mu^2 + \mu
\end{align*}
\end{solution}

\include{ch6}
\pagebreak
\begin{problem}[Exercise 7.1]
    Let \( W \) be a Brownian motion and let \( \mathbb{F} = (\mathcal{F}_t)_{t\geq0} \) be a filtration for \( W \). Show that \( W(t)^2 - t \) is a martingale with respect to the filtration \( \mathbb{F} \).
\end{problem}

\begin{solution}[Solution]
Suppose \( X\sim \mathcal{N}(0,\sigma^2) \). Then,
\begin{align*}
    \sigma^2 = \mathbb{V} \left[ X \right] = \EE[X^2] - \EE[X]^2 = \EE[X^2] - 0^2 = \EE[X^2]
\end{align*}

Let \( 0 \leq s \leq t \). 
By the definition of a filtration, \( (W(t)-W(s)) \) is independent of \( \mathcal{F}_s \). Moreover, by the definition of Brownian Motion we have \( W(t)-W(s) \sim \mathcal{N}(0,t-s) \). Thus,
\begin{align*}
    \EE \left[ (W(t)-W(s))^2 \big| \mathcal{F}_s \right] = \EE \left[ (W(t) - W(s))^2 \right] = (t-s)
\end{align*}

Since \( W(s) \in \mathcal{F}_s \), by ``taking out what is known'' we have,
\begin{align*}
    \EE \left[ W(t)W(s) \big|\mathcal{F}_s \right] 
    = W(s) \EE\left[ W(t) \big| \mathcal{F}_s \right]
    = W(s)W(s)
    = W(s)^2 
    \\
    \EE \left[ W(s)^2 \big|\mathcal{F}_2 \right]
    = W(s) \EE \left[ W(s) \big|\mathcal{F}_2 \right]
    = W(s)W(s)
    = W(s)^2
\end{align*}

Therefore,
\begin{align*}
    \EE \left[ W(t)^2 - t \big| \mathcal{F}_s \right] 
    &= \EE \left[ (W(t)-W(s)+W(s))^2 -t \right] 
    \\ &= \EE \left[ (W(t)-W(s))^2 + 2(W(t)-W(s))W(s)+W(s)^2-t \right]
    \\ &= \EE \left[ (W(t)-W(s))^2 \big| \mathcal{F}_s \right] + 2 \EE \left[ W(t)W(s) \big| \mathcal{F}_s \right] - \EE \left[ W(s)^2 \big| \mathcal{F}_2 \right] - \EE \left[ t \right]
    \\ &= (t-s) + 2 W(s)^2 - W(s)^2-t
    \\ &= W(s)^2 - s
\end{align*}

This proves \( W(t) - t \) is a martingale with respect to the filtration \( \mathbb{F} \). \qed
\end{solution}

\pagebreak
\begin{problem}[Exercise 7.2]
    Compute the characteristic function of \( W(N(t)) \) where \( N \) is a Poisson process with intensity \( \lambda \) and the Brownian motion \( W \) is independent of the Poisson process \( N \).
\end{problem}

\begin{solution}[Solution]
The characteristic function is defined as,
\begin{align*}
    \phi(s) = \EE e^{isW(N(t))}
\end{align*}

We condition on \( N(t) \) using iterated conditioning,
\begin{align*}
    \EE \left[ e^{is W(N(t))} \right] = 
    \EE \left[\EE\left[e^{is W(N(t))} \Big| N(t) \right] \right]
\end{align*}

The characteristic function of \( Z\sim\mathcal{N}(\mu,\sigma^2) \) is \( \phi_Z(s) = \exp(i\mu s-\sigma^2s^2/2) \).
At time \( t \), \( W(t) \) is normally distributed with mean zero and variance \( t \). Thus,
\begin{align*}
    \EE \left[\EE\left[e^{is W(N(t))} \Big| N(t) \right] \right] =
    \EE \left[ e^{ -N(t)s^2 /2}\right]
\end{align*}

Since \( N(t) \) is a Poisson process with parameter \( \lambda \), then \( N(t) = k \) with probability \( (\lambda t)^ke^{-\lambda t}/k! \). Thus,
\begin{align*}
    \EE \left[ e^{ -N(t)s^2 /2}\right]
    \sum_{k=0}^{\infty} \dfrac{(\lambda t)^k}{k!}e^{-\lambda t} e^{-ks^2/2} = 
    e^{- \lambda t}\sum_{k=0}^{\infty} \dfrac{(\lambda t)^k}{k!} \left( e^{-s^2/2} \right)^k 
\end{align*}

Simplifying yields,
\begin{align*}
    e^{- \lambda t}\sum_{k=0}^{\infty} \dfrac{(\lambda t)^k}{k!} \left(  e^{-s^2/2} \right)^k =
    e^{-\lambda t}\sum_{k=0}^{\infty} \dfrac{1}{k!}\left(\lambda t e^{-s^2/2} \right)^k = 
    e^{-\lambda t} \exp \left( \lambda t e^{-s^2/2} \right) =
    \exp \left( \lambda t \left( e^{-s^2/2}-1 \right) \right)
\end{align*}

That is, the characteristic function \( \phi(s) \) of \( W(N(t)) \) is,
\begin{align*}
    \phi(s) = \exp \left( \lambda t \left( e^{-s^2/2}-1 \right) \right)
\end{align*}
\end{solution}


\pagebreak
\begin{problem}[Exercise 7.3]
    The \( n \)-th variation of a function \( f \), over the interval \( [0,T] \) is defined as,
    \begin{align*}
        V_T(n,f) := 
        \lim_{\norm{\Pi}\to 0} \sum_{j=0}^{m-1} |f(t_{j+1})-f(t_j)|^n, && \Pi = \{0=t_0,t_1, \ldots, t_m=T\}, && \norm{\Pi} = \max_j(t_{j+1}-t_{j})
    \end{align*}

    Show that \( V_T(1,W) = \infty \) and \( V_T(3,W) = 0 \), where \( W \) is a Brownian motion.
\end{problem}


\begin{solution}[Solution]
We first prove the following useful lemma.
\begin{lemma}
    If \( f_n \to 0 \) and \( |g_n| \leq M \) for some \( |M| < \infty \) then \( (f_ng_n)\to 0 \).
\end{lemma}
Fix, \( \varepsilon > 0 \). Then, by convergence of \( f_n \) there is some \( N\in\NN \) such that \( |f_n| < \varepsilon/M \) for all \( n\geq N \). Then,
\begin{align*}
    |f_ng_n| = |f_n||g_n| \leq |f_n|M < (\varepsilon/M)M = \varepsilon
\end{align*}

This proves \( f_ng_n \to 0 \). \qed


Write,
\begin{align*}
    V_T(k+1,W) = 
    \lim_{\norm{\Pi}\to 0} \sum_{j=0}^{m-1} |W(t_{j+1})-W(t_j)|^{k+1} = 
    \lim_{\norm{\Pi}\to 0} \sum_{j=0}^{m-1} |W(t_{j+1})-W(t_j)|^k|W(t_{j+1})-W(t_j)|
\end{align*}

Let, \( M_\Pi = \max_j |W(t_{j+1}) - W(t_j)| \) for a given partition \( \Pi \). Then,
\begin{align*}
    \lim_{\norm{\Pi}\to 0} \sum_{j=0}^{m-1} |W(t_{j+1})-W(t_j)|^k|W(t_{j+1})-W(t_j)| 
    &\leq \lim_{\norm{\Pi}\to 0} \sum_{j=0}^{m-1} |W(t_{j+1})-W(t_j)|^k M_\Pi 
    \\&= \lim_{\norm{\Pi}\to 0} M_\Pi \sum_{j=0}^{n-1} |W(t_{j+1})-W(t_j)|^k
\end{align*}

Provided, \( |V_T(k,T)| = V_T(k,T) \) is not infinite,
\begin{align*}
    \lim_{\norm{\Pi}\to 0} M_\Pi \sum_{j=0}^{m-1} |W(t_{j+1})-W(t_j)|^k
    = \left( \lim_{\norm{\Pi}\to 0} M_\Pi \right) \left( \lim_{\norm{\Pi}\to 0} \sum_{j=0}^{n-1} |W(t_{j+1})-W(t_j)|^2 \right)
\end{align*}

Since \( W(t) \) is continuous, \( |W(t_{j+1}) - W(t_j)| \to 0 \) as \( \norm{\Pi}\to 0 \) since \( t_{j+1} - t_{j} \to 0 \). In particular, this means that \( M_\Pi \to 0 \) as \( \norm{\Pi} \to 0 \).

Thus,
\begin{align*}
    0 \geq V_T(k+1,W) = 
    \left( \lim_{\norm{\Pi}\to 0} M_\Pi \right) \left( \lim_{\norm{\Pi}\to 0} \sum_{j=0}^{m-1} |W(t_{j+1})-W(t_j)|^k \right) \leq 
    0\cdot N 
    = 0
\end{align*}

Recall \( V_T(2,W) = T < \infty \). 
Then, by above, \( V_T(3,W) = 0 \). \qed

Suppose, for the sake of contradiction that \( V_T(1,W) \neq \infty \). Clearly \( V_T(1,W) \geq 0 \), so \( V_T(1,W) \) is bounded above and below by finite constants. Then, by above, \( V_T(2,W) = 0 \), a contradiction (for \( T>0 \)). This proves \( V_T(1,W) = \infty \). \qed 
\end{solution}


\pagebreak
\begin{problem}[Exercise 7.4]
Define
\begin{align*}
    X_t = \mu t+W_t && \tau_m:=\inf\{t\geq 0:X_t=m\}
\end{align*}
Show that \( Z \) is a martingale where,
\begin{align*}
    Z_t = \exp(\sigma X_t-(\sigma \mu+\sigma^2/2)t)
\end{align*}

    Assume \( \mu>0 \) and \( m\geq 0 \). Assume further that \( \tau_m < \infty \) with probability one and the stopped process \( Z_{t\wedge \tau_m} \) is a martingale. Find the Laplace transform \( \EE e^{-\alpha \tau_m} \).
\end{problem}


\begin{solution}[Solution]
Let \( 0\leq s\leq t \). Rewrite,
\begin{align*}
    \EE \left[ Z_t \big| \mathcal{F}_s \right]
    = \EE \left[ e^{\sigma X_t - (\sigma \mu+\sigma^2/2)t } \big|\mathcal{F}_s \right]
    = \EE \left[ e^{\sigma (\mu t+W_t) - (\sigma \mu+\sigma^2/2)t } \big|\mathcal{F}_s \right]
    = \EE \left[ e^{\sigma W_t - (\sigma^2/2)t } \big|\mathcal{F}_s \right]
\end{align*}

Now, pulling out what is known,
\begin{align*}
    \EE \left[ e^{\sigma W_t - (\sigma^2/2)t } \big|\mathcal{F}_s \right]
    = \EE \left[ e^{\sigma (W_t-W_s) + \sigma W_s-(\sigma^2/2)t)} \big|\mathcal{F}_s \right]
    = e^{\sigma W_s - (\sigma^2/2)t} \EE \left[ e^{\sigma (W_t-W_s)} \big|\mathcal{F}_s \right]
\end{align*}

By the property of independent increments,
\begin{align*}
    e^{\sigma W_s - (\sigma^2/2)t} \EE \left[ e^{\sigma (W_t-W_s)} \big|\mathcal{F}_s \right]
    = e^{\sigma W_s - (\sigma^2/2)t} \EE \left[ e^{\sigma (W_t-W_s)} \right]
    = e^{\sigma W_s - (\sigma^2/2)t} e^{\sigma^2(t-s)/2}
\end{align*}

Finally,
\begin{align*}
    e^{\sigma W_s - (\sigma^2/2)t} e^{\sigma^2(t-s)/2}
    =e^{\sigma W_s - (\sigma^2/2)s}
    =e^{\sigma (\mu s+W_s) - (\sigma\mu  +\sigma^2/2)s}
    =e^{\sigma X_2 - (\sigma\mu  +\sigma^2/2)s}
\end{align*}

This proves \( Z_t \) is a martingale. \qed


Define \( s=\min\{t,\tau_m\} \). Fix \( m\geq 0 \) and define,
\begin{align*}
    Z^{(m)} = \left( Z_t^{(m)} \right)_{t\geq 0}, && Z_t^{(m)} = Z_s
\end{align*}

Then, using the fact that \( Z_t \) is a martingale we have,
\begin{align*}
    1 = Z_0^{(m)} = \EE \left[ Z_t^{(m)} \right]
    = \EE \left[ e^{\sigma X_s - (\sigma \mu+\sigma^2/2) s} \right]
\end{align*}

If \( \tau_m = \infty \) then \( X_{t} < m \) for all \( t \). Thus, since \( \sigma\geq0, \mu>0 \), 
\begin{align*}
    e^{\sigma X_t - (\sigma\mu+\sigma^2/2)t} \leq  
    e^{\sigma m - (\sigma\mu+\sigma^2/2)t} < \infty
\end{align*}

Therefore, since \( \PP(\tau_m <\infty) = 0 \),
\begin{align*}
    \EE \left[ e^{\sigma X_s - (\sigma \mu+\sigma^2/2) s} \right]
    &= \EE \left[ \mathbbm{1}_{\{\tau_m = \infty\}} \left( e^{\sigma X_s - (\sigma \mu+\sigma^2/2) s} \right)+ \mathbbm{1}_{\{\tau_m < \infty\}} \left( e^{\sigma X_s - (\sigma \mu+\sigma^2/2) s} \right) \right]
    \\&= \EE \left[ \mathbbm{1}_{\{\tau_m = \infty \}} \left( e^{\sigma X_t - (\sigma \mu+\sigma^2/2) t} \right) \right] + \EE \left[ \mathbbm{1}_{\{\tau_m<\infty\}} \left( e^{\sigma X_{\tau_m} - (\sigma \mu+\sigma^2/2) \tau_m} \right) \right]
    \\&= 0 + \EE \left[ \mathbbm{1}_{\{\tau_m<\infty\}} \left( e^{\sigma m - (\sigma \mu+\sigma^2/2) \tau_m} \right) \right]
\end{align*}

Similarly, since \( \sigma\geq 0, \mu>0 \), \( e^{\sigma m - (\sigma\mu+\sigma^2/2)\tau_m) } < \infty  \). Therefore,
\begin{align*}
    \EE \left[ \mathbbm{1}_{\{\tau_m<\infty\}} \left( e^{\sigma m - (\sigma \mu+\sigma^2/2) \tau_m} \right) \right]
    &= \EE \left[ \mathbbm{1}_{\{\tau_m=\infty\}} \left( e^{\sigma m - (\sigma \mu+\sigma^2/2) \tau_m} \right) \right]
    +\EE \left[ \mathbbm{1}_{\{\tau_m<\infty\}} \left( e^{\sigma m - (\sigma \mu+\sigma^2/2) \tau_m} \right) \right]
    \\&= \EE \left[ \mathbbm{1}_{\{\tau_m=\infty\}} \left( e^{\sigma m - (\sigma \mu+\sigma^2/2) \tau_m} \right) + \mathbbm{1}_{\{\tau_m<\infty\}} \left( e^{\sigma m - (\sigma \mu+\sigma^2/2) \tau_m} \right) \right]
    \\&= \EE \left[ e^{\sigma m - (\sigma \mu+\sigma^2/2) \tau_m} \right]
\end{align*}

\iffalse
In the limit \( t\to\infty \), since \( \tau_m < \infty\) a.s. , \( s = \min\{t,\tau_m\} \to \tau_m < \infty \) and \( X_s \to X_{\tau_m} = m \). Thus,
\begin{align*}
    1 = \lim_{t\to\infty} \EE \left[ e^{\sigma X_s - (\sigma \mu+\sigma^2/2) s} \right]
      = \EE \left[ \lim_{t\to\infty} e^{\sigma X_s - (\sigma \mu+\sigma^2/2) s} \right]
      = \EE \left[ e^{\sigma m - (\sigma \mu+\sigma^2/2) \tau_m} \right]
\end{align*}
\fi

Then, setting \( \alpha = (\sigma\mu+\sigma^2/2) \),
\begin{align*}
    e^{-\sigma m} = \EE \left[ e^{-(\sigma \mu+\sigma^2/2)\tau_m} \right] = \EE \left[ e^{-\alpha \tau_m} \right]
\end{align*}

We solve the equation, \( \alpha = (\sigma\mu + \sigma^2/2) \) for \( \sigma \) using the quadratic equation, yielding,
\begin{align*}
    \sigma 
    %= \left( -\mu \pm \sqrt{\mu^2-4(1/2)(-\alpha)} \right)/\left(2 (1/2) \right)
    = -\mu \pm \sqrt{\mu^2+2\alpha}
\end{align*}

However, \( \sigma,\alpha \geq 0 \) so we must take \( \sigma = -\mu + \sqrt{\mu^2+2\alpha} \).
Thus,
\begin{align*}
    \EE \left[ e^{-\alpha \tau_m} \right] = e^{\left(\mu-\sqrt{\mu^2+2\alpha}\right)m}
\end{align*}

\end{solution}


\begin{problem}[Exercise 8.1]
    Compute \( \d(W_t^4) \). Write \( W_T^4 \) as an integral with respect to \( W \) plus an integral with respect to \( t \). Use this representation of \( W_T^4 \) to show that \( \EE W_T^4 = 3T^2 \). Compute \( \EE W_T^6 \) using the same technique.
\end{problem}

\begin{solution}[Solution]
Write \( f(x) = x^4 \) so that \( f(W_t) = W_t^4 \). Then, \( f'(x) = 4x^3 \) and \( f''(x) = 12x^2 \). Therefore, It\^o's formula gives,
\begin{align*}
    \d W_t^4 &= f'(W_t)\d W_t + \frac{1}{2}f''(W_t) \d[W,W]_t 
    = 4W_t^3 \d W_t + \frac{12}{2} W_t^2 \d[W,W]_t 
\end{align*}

Thus, writing \( \d[W,W]_t = \d t \) we have,
\begin{align*}
    \d W_t^4 = 4W_t^3 \d W_t + 6W_t^2 \d t
\end{align*}
    
Thus, since \( W_0 = 0 \),
\begin{align*}
    W_T^4 = W_T^4 - W_0^4 = 4\int_{0}^{T} W_t^3\d W_t + 6 \int_{0}^{T}W_t^2 \d t
\end{align*}

Recall It\^o integrals are martingales so that,
\begin{align*}
    \EE \left[ \int_{0}^{T}W_t^3\d W_t \right] = 0
    %\EE \left[ \EE \left[ \int_{0}^{T}W_t^3 \d W_t \Bigg| \mathcal{F}_T \right] \right] = \EE \left[ \int_{0}^{0}W_t^3 \d W_t \right] = 0
\end{align*}

Note also that since \( \EE \left[ W_t^2 \right] = t \),
\begin{align*}
    \EE \left[ \int_{0}^{T}W_t^2 \d t \right] = \int_{0}^{T}\EE \left[ W_t^2 \right]\d t = \int_{0}^{T} t \d t = \dfrac{T^2}{2}
\end{align*}

Therefore,
\begin{align*}
    \EE \left[ W_T^4 \right] &= 4\EE \left[ \int_{0}^{T}W_t^3 \d W_t \right] + 6 \EE \left[ \int_{0}^{T}W_t^2 \d t \right] = 6\dfrac{T^2}{2} = 3 T^2
\end{align*}

Similarly, we have,
\begin{align*}
    W_T^6 = 6\int_{0}^{T} W_t^5 \d W_t + \dfrac{6\cdot 5}{2}\int_{0}^{T} W_t^4 \d t 
    %= 30 \int_{0}^{T} \dfrac{t^3}{3} \d t = 30 \dfrac{T^4}{12} = \dfrac{5}{2}T^4 
\end{align*}

Therefore, since \( \EE \left[ W_t^4 \right] = 3t^2 \),
\begin{align*}
    \EE\left[W_T^6\right] = 6 \EE\left[\int_{0}^{T} W_t^5 \d W_t\right] + 15 \EE\left[\int_{0}^{T} W_t^4 \d t \right] = 15 \int_{0}^{T} \EE \left[ W_t^4 \right] \d t = 15 \int_{0}^{T} 3t^2 \d t = 15T^3 
\end{align*}
\end{solution}

\begin{problem}[Exercise 8.2]
Find an explicit expression for \( Y_T \) where,
\begin{align*}
    \d Y_t = r \d t + \alpha Y_t \d W_t
\end{align*}

    Hint: Multiply the above equation by \( F_t := \exp(- \alpha W_t + \frac{1}{2} \alpha^2t) \).

\end{problem}


\begin{solution}[Solution]
Let \( f(x,y) = \exp(-\alpha x + \frac{1}{2} \lambda^2 y) \) so that,
\begin{align*}
    f_x(W_t,t) = -\alpha F_t && f_y(W_t,t) = \frac{\alpha^2}{2} F_t && f_{xx}(W_t,t) =  \alpha^2 F_t
\end{align*}

Then \( F_t = f(W_t,t) \), so by It\^o's formula and the heuristic \( (\d W_t)^2 =\d t, (\d t)^2 = \d t \d W_t = 0 \),
\begin{align*}
    \d F_t = \d f(W_t,t) &= f_y(W_t,t) \d t +  f_x(W_t,t) \d W_t + \frac{1}{2} f_{xx}(W_t,t) (\d W_t)^2 
    \\&= \frac{\alpha^2}{2}  F_t \d t -\alpha F_t \d W_t + \frac{\alpha^2}{2} F_t \d t
    \\&= \alpha^2 F_t \d t- \alpha F_t \d W_t
\end{align*}


Using our heuristics we have,
\begin{align*}
    \d[F,Y]_t = (\d F_t)(\d Y_t) = \left( \alpha^2 F_t \d t - \alpha F_t \d W_t \right) \left( r \d t + \alpha Y_t \d W_t \right)
    = -\alpha^2 F_tY_t (\d W_t)^2 
    = -\alpha^2 F_tY_t \d t
\end{align*}


By the product rule we have,
\begin{align*}
    \d (F_tY_t) &=  F_t \d Y_t + Y_t \d F_t + \d[F,Y]_t
    \\&= F_t (r\d t+\alpha Y_t \d W_t) + Y_t (\alpha^2 F_t \d t-\alpha F_t \d W_t) - \alpha^2 F_t Y_t \d t
    \\&= r F_t \d t 
\end{align*}

In integral form,
\begin{align*}
    F_t Y_t - F_0Y_0 = \int_{0}^{t}r F_s \d s = \int_{0}^{t} r e^{-\alpha W_s + \frac{1}{2} \alpha^2 s} \d s
\end{align*}

We can add \( F_0Y_0 = Y_0 \) and divide by \( F_t \) yielding,
\begin{align*}
    Y_t = Y_0 + r e^{\alpha W_t - \frac{1}{2}\alpha^2 t} \int_{0}^{t} e^{-\alpha W_s + \frac{1}{2} \alpha^2 s} \d s
\end{align*}
\end{solution}

\begin{problem}[Exercise 8.3]
Suppose \( X \), \( \Delta \), and \( \Pi \) are given by,
\begin{align*}
    \d X_t = \sigma X_t \d W_t, 
    && \Delta_t = \pp{f}{x} (t,X_t),
    && \Pi_t = X_t \Delta_t
\end{align*}
where \( f \) is some smooth function. Show that if \( f \) satisfies,
\begin{align*}
    \left( \pp{}{t} + \dfrac{1}{2}\sigma^2x^2 \pp[2]{}{x} \right) f(t,x) = 0
\end{align*}
    for all \( (t,x) \), then \( \Pi \) is a martingale with respect to a filtration \( \mathcal{F}_t \) for \( W \).
\end{problem}


\begin{solution}[Solution]
We have,
\begin{align*}
    \pp{}{x} \left( \pp{}{t} + \dfrac{1}{2}\sigma^2x^2 \pp[2]{}{x}\right)  
    = \dfrac{\partial^2}{\partial x\partial t} + \dfrac{1}{2}\sigma^2 \left[ x^2 \pp[3]{}{x} + 2x \pp[2]{}{x} \right]
\end{align*}

Thus, using the condition for \( f \) we have,
\begin{align*}
    \dfrac{\partial^2 f}{\partial x\partial t} + \dfrac{1}{2}\sigma^2 X_t^2 \pp[3]{f}{x} = - \sigma^2 X_t \pp[2]{f}{x}
\end{align*}

Using our heuristics we have,
\begin{align*}
    \d[X,X] = \sigma^2 X_t^2 (\d W_t)^2 = \sigma^2 X_t^2 \d t
\end{align*}

Similarly, 
\begin{align*}
    \d[X,t] = \d[t,X] = \d[t,t] = 0
\end{align*}

Therefore, by It\^o's formula,
\begin{align*}
%    \d \Delta_t &=  F_x(t,X_t) \d t + F_t(t,X_t) \d X_t + \dfrac{1}{2} F_{xx}(t,X_t) \d [X,X]
    \d \Delta_t &= \dfrac{\partial^2 f}{\partial x\partial t}(t,X_t) \d t + \pp[2]{f}{x}(t,X_t) \d X_t + \dfrac{1}{2} \d[X,X]
    \\&=  \dfrac{\partial^2 f}{\partial x\partial t}(t,X_t) \d t + \sigma X_t \pp[2]{f}{x}(t,X_t) \d W_t + \dfrac{1}{2}\sigma^2X_t^2 \pp[3]{f}{x}(t,X_t) \d t
    \\&= -\sigma^2 X_t \pp[2]{f}{x}(t,X_t) \d t + \sigma X_t \pp[2]{f}{x}(t,X_t)\d W_t 
\end{align*}

Therefore,
\begin{align*}
    \d[X,\Delta]_t = (\d X_t)(\d \Delta_t) 
    = \sigma^2 X_t^2 \dfrac{\partial^2f}{\partial x^2}(t,X_t) (\d W_t)^2 
    = \sigma^2 X_t^2 \dfrac{\partial^2f}{\partial x^2}(t,X_t) \d t
\end{align*}


Finally, we have,
\begin{align*}
    \d \Pi_t &= \d(X_t \Delta _t) = X_t \d\Delta _t + \Delta_t \d X_t + \d[X,\Delta]_t \\
    &= X_t \left( -\sigma^2 X_t \pp[2]{f}{x}(t,X_t) \d t + \sigma X_t \pp[2]{f}{x}(t,X_t)\d W_t \right) 
    + \sigma X_t \pp{f}{x}(t,X_t) \d W_t 
    + \sigma^2 X_t^2 \pp[2]{f}{x} \d t 
    \\&= \sigma X_t \left( X_t \pp[2]{f}{x}(t,X_t) + \pp{f}{x}(t,X_t)\right) \d W_t
\end{align*}

Since there is no \( \d t \) dependence this is an It\^o integral and therefore a martingale with respect to a filtration for \( W \). (there are probably some technical assumptions we need about \( X \) and \( f \), but in class we never dealt with these)\qed

\end{solution}

\begin{problem}[Exercise 8.4]
Suppose \( X \) is given by,
\begin{align*}
    \d X_t = \mu(t,X_t)dt + \sigma(t,X_t) \d W_t
\end{align*}
For any smooth function \( f \) define,
\begin{align*}
    M_t^f:= f(t,X_t) - f(0,X_0) - \int_{0}^{t} \left( \pp{}{s} + \mu(s,X_s)\pp{}{x} + \dfrac{1}{2}\sigma^2(s,X_s) \pp[2]{}{x} \right) f(s,X_s)ds
\end{align*}
    Show that \( M^f \) is a martingale with respect to a filtration \( \mathcal{F}_t \) for \( W \).
\end{problem}


\begin{solution}[Solution]
We first compute,
\begin{align*}
    \d[X,X]_t = (\d X_t)(\d X_t) = \sigma^2(t,X_t) (\d W_t)^2 = \sigma^2(t,X_t)\d t
\end{align*}

We then have,
\begin{align*}
    \d f(t,X_t) &= \pp{f}{t}(t,X_t) \d t + \pp{f}{x}(t,X_t) \d X_t + \dfrac{1}{2}\pp[2]{f}{x}\d[X,X]_t 
    \\&= \pp{f}{t}(t,X_t) \d t + \pp{f}{x}(t,X_t)  [\mu(t,X_t)\d t +  \sigma(t,X_t)\d W_t] + \dfrac{1}{2}\sigma^2(t,X_t) \pp[2]{f}{x} \d t 
    \\&= \left( \pp{}{t} + \mu(t,X_t) \pp{}{x} + \dfrac{1}{2}\sigma^2(t,X_t) \pp[2]{}{x} \right) f(t,X_t) \d t + \sigma(t,X_t) \pp{f}{x} \d W_t 
\end{align*}

Finally, since \( f(0,X_0) \) is a constant,
\begin{align*}
    \d M_t^f &= \d f(t,X_t) - \left( \pp{}{t} + \mu(t,X_t)\pp{}{x} + \dfrac{1}{2}\sigma^2(t,X_t) \pp[2]{}{x} \right) f(t,X_t) \d t \\
    &= \sigma(t,X_t) \pp{f}{x} \d W_t 
\end{align*}

Since there is no \( \d t \) dependence this an It\^o integral and therefore a martingale with respect to a filtration for \( W \). \qed
\end{solution}


\pagebreak

\pagebreak




\end{document}
