\begin{problem}[Exercise 4.1]
A six-sided die is rolled repeatedly. Which of the following a Markov chains? For those that are, find the one-step transition matrix. 
\begin{enumerate}[nolistsep,label=(\alph*)]
	\item \( X_n \) is the largest number rolled up to the nth roll. 
	\item \( X_n \) is the number of sixes rolled in the first \( n \) rolls. 
	\item At time \( n \), \( X_n \) is the time since the last six was rolled.
	\item At time \( n \), \( X_n \) is the time until the next six is rolled.
\end{enumerate}
\end{problem}

\begin{solution}[Solution]
\begin{enumerate}[label=(\alph*)]
	\item Yes.
        \begin{align*}
            P = 
            \left[\begin{array}{cccccc}
                1/6 & 1/6 & 1/6 & 1/6 & 1/6 & 1/6 \\
                & 2/6 & 1/6 & 1/6 & 1/6 & 1/6 \\
                && 36 & 1/6 & 1/6 & 1/6 \\
                &&& 4/6 & 1/6 & 1/6 \\
                &&&& 5/6 & 1/6 \\
                &&&& & 1
            \end{array}\right]
        \end{align*}
        
	\item Yes.
        \begin{align*}
            P=
            \left[\begin{array}{cccc}
                5/6 & 1/6 \\
                & 5/6 & 1/6 \\
                && \ddots & \ddots
            \end{array}\right]
        \end{align*}
        
    \item Yes. Suppose \( X_n = i \). The next roll is either a 6, in which case \( X_{n+1} = 0 \). Otherwise \( X_{n+1} = i+1 \).  
        \begin{align*}
            P = \left[\begin{array}{ccccc}
            1/6 & 5/6 \\
            1/6 & & 5/6 \\
            1/6 &  & & 5/6 \\
            \vdots & & & & \ddots
            \end{array}\right]
        \end{align*}
        
    \item Yes. Suppose \( X_n = 0 \). The probability of \( X_{n+1} = j \) is \( (1/6)(5/6)^j \) as you must not roll a 6 for \( j \) turns, and then must roll a \( 6 \) on the \( j \)-th. Suppose \( X_n = i > 0 \). Then the next step you will be on turn closer to rolling a 6. That is, \( X_{n+1} = i-1 \).
        \begin{align*}
            P =
            \left[\begin{array}{cccccc}
                \frac{1}{6} & \frac{1}{6}\left(\frac{5}{6}\right) & \frac{1}{6} \left( \frac{5}{6}^2 \right) & \frac{1}{6} \left( \frac{5}{6} \right)^3 & \cdots \\
                1 \\
                & 1 \\
                & & 1 \\
                & & & 1 \\
                & & & & \ddots
            \end{array}\right]
        \end{align*}
        
\end{enumerate}
\end{solution}

\begin{problem}[Exercise 4.2]
    Let \(Y_n=X_{2n} \). Compute the transition matrix for \( Y \) when 
    \begin{enumerate}[nolistsep,label=(\alph*)]
        \item \( X \) is a simple random walk (i.e., \( X \) increases by one with probability \( p \) and decreases by 1 with probability \( q \))
        \item \( X \) is a branching process where \( G \) is the generating function of the number of offspring from each individual
    \end{enumerate}
\end{problem}

\begin{solution}[Solution]
\begin{enumerate}[label=(\alph*)]
    \item
        In each step we can go down with probability \( q \) and then down again with probability \( q \) or up with probability \( p \). Alternatively we can go up with probability \( p \) and then down with probability \( q \) or up again with probability \( p \). 

        Therefore we will end up two spaces down with probability \( q^2 \), in the same position with probability \( qp+pq = 2pq \), or up two spaces with probability \( p^2 \). Thus,
        \begin{align*}
            p(i,j) = \begin{cases} p^2 & j=i+2 \\ 2pq & i=j \\ q^2 & j=i-2 \\ 0 & \text{otherwise}\end{cases} 
        \end{align*}
    \item
    We can obtain the exponents of a generating function \( G(s) = a_0 + a_1s+a_2s^2 + ... \) by,
        \begin{align*}
            a_n = \dfrac{1}{n!} \dfrac{d^n}{ds^n}\Big[G(s) \Big]_{s=0}
        \end{align*}

        The coefficient of the \( s^k \) term is the value of the probability mass function of \( X \) evaluated at \( k \).

        The generating function of \( Y \) is \( G(G(s)) = G_2(s) \) from the notes.

        For a branching process with current population \( k \), the population of the next generation will be \( X_1+X_2+...+X_k\), where each \( X_i \) is iid with distribution \( X \). Therefore,    
        \begin{align*}
            p(i,j) = \dfrac{1}{j!} \dfrac{d^n}{ds^n}\Big[ G_2(s)^i \Big]_{s=0}
        \end{align*}
\end{enumerate}
\end{solution}

\begin{problem}[Exercise 4.3]
Let \( X \) be a Markov chain with state space \( S \) and absorbing state \( k \) (i.e., \( p(k, j) = 0 \) for all \( j \in S \)). Suppose \( j\rightarrow k\) for all \( j\in S \). Show that all states other than \( k \) are transient.
\end{problem}

\begin{solution}[Solution]
Fix a state \( j\in S \). By definition of \( j\rightarrow k \), \( \exists N\geq 0 : p_N(j,k) \geq 0\). Since \( \{ X_N = k | X_0 = j \} \subseteq \{\forall n, X_n \neq j | X_0 = j \} \) we have,
\begin{align*}
    0 < p_N(j,k) = \PP(X_N = k | X_0 = j) \leq \PP(\forall n, X_n\neq j | X_0 = j)
\end{align*}

Therefore,
\begin{align*}
    \PP(\exists n\geq 0: X_n = j | X_0 = j) = 1 - \PP(\forall n, X_n \neq j|X_0 = j) < 1
\end{align*}

This proves state \( j \) istransient. \qed


\end{solution}

\begin{problem}[Exercise 4.4]
Suppose two distinct states \( i,j \) satisfy
\begin{align*}
    \PP(\tau_j<\tau_i | X_0 = i ) = \PP(\tau_i < \tau_j | X_0 = j)
\end{align*}
    where \( \tau_j = \inf\{ n\geq 1 : X_n = j \} \). Show that, if \( X_0=i \), the expected value of visits to \( j \) prior to returning to \( i \) is one.
\end{problem}

\begin{solution}[Solution]
Write
\begin{align*}
    p=\PP(\tau_j<\tau_i | X_0 = i ) = \PP(\tau_i < \tau_j | X_0 = j)
\end{align*}

That is,
\( p \) is the probability that we go to state \( j \) before state \( i \) give we are in state \( i \), and \( p \) is also the probability that we go to state \( i \) before state \( j \) given we are in state \( j \).

Then \( 1-p \) is the probability that we do not go to state \( i \) before returning state \( j \),0 given we start in state \( j \).

So \( (1-p)^k \) is the probability that we return to state \( j \) exactly \( k \) times before moving to state \( i \), given we start in state \( j \).

Let \( N \) be the number of visits to \( j \) prior to returning to \( i \) given we start in state \( i \). 

The probability that \( N = k\in\ZZ_{\geq 0} \) is the probability that starting from state \( i \) we go to state \( j \), return to state \( j \) \( (k-1) \) times without returning to state \( i \), and then return to state \( i \) without going to returning to state \( j \).

So \( \PP(N=k | X_0 = i) = p(1-p)^{k-1}p\). This is the probability mass function for \( N \) so,
\begin{align*}
    \EE[N] = \sum_{n=0}^{\infty} np^2(1-p)^{k-1} = p \sum_{n=0}^{\infty}n(1-p)^n = p \dfrac{p}{(1-(1-p))^2} = 1
\end{align*}
\end{solution}

\begin{problem}[Exercise 4.5]
Let X be a Markov chain with transition matrix,
\begin{align*}
    P = \left[\begin{array}{ccc}1-2p & 2p & 0 \\ p & 1-2p & p \\ 0 & 2p & 1-2p\end{array}\right], && p\in(0,1)
\end{align*}
    Find \( P^n \), the invariant distribution \( \pi \), and the mean-recurrence times \( \overline{\tau}_j \) for \( j=1,2,3 \).
\end{problem}

\begin{solution}[Solution]
    Note that \( P \) has eigendecomposition \( P = V\Lambda V^{-1} \) where,
    \begin{align*}
        \Lambda =
        \left[\begin{array}{rrr}1 \\ &1-4p \\ && 1-2p\end{array}\right]
        ,&&
        V = \left[\begin{array}{rrr} 1 & 1 & -1 \\ 1 & -1 & 0 \\ 1 & 1 & 1\end{array}\right]
    \end{align*}
    
    Therefore, \( P^n = V\Lambda^n V^{-1} \). Explicitly,
    \begin{align*}
        P^n = 
        \left[\begin{array}{rrr} 1 & 1 & -1 \\ 1 & -1 & 0 \\ 1 & 1 & 1\end{array}\right]
        \left[\begin{array}{rrr}1 \\ &1-4p \\ && 1-2p\end{array}\right]
        \left[\begin{array}{rrr}1/4 & 1/2 & 1/4 \\ 1/4 & -1/2 & 1/4 \\ -1/2 & 0 & 1/2\end{array}\right]
    \end{align*}
    
    Invariant distributions are linear combinations of left eigenvectors corresponding to eigenvalues of 1. In this case that is the first row of \( V^{-1} \). That is,
    \begin{align*}
        \pi = \left[\begin{array}{rrr}\frac{1}{4} & \frac{1}{2} & \\frac{1}{4}\end{array}\right]
    \end{align*}

    Finally, since the invariant distribution is unique, by Theorem we have,
    \begin{align*}
        \overline{\tau}_i = \frac{1}{\pi(i)}
    \end{align*}
    
    
\end{solution}


\begin{problem}[Exercise 4.6]
    Let \( X_n \) be the number of mistakes in the \( n \)-th addition of a book. Between the \( n \)-th and the \( (n+1) \)-th addition an editor corrects each mistake independently with probability \( p \) and introduces \( Y_n \) new mistakes where the\( (Y_n) \) are iid and Poisson distributed with parameter \( \lambda \). Find the invariant distribution \( \pi \) of the number of mistakes in the book. 
\end{problem}

\begin{solution}[Solution]
Let \( M_{n,k} \) be distributed as \( \operatorname{Ber}(1-p) \) so that \( M_k \) is 0 if this mistake is corrected, and 1 otherwise. Let \( Y_n \) be Poisson distributed with parameter \( \lambda \). Then,
\begin{align*}
    X_{n+1} = Y_n + \sum_{k-1}^{X_n} M_k 
\end{align*}

Each \( M_{n,k} \) has generating function,
 \begin{align*}
     G_{M_{n,k}} = p+(1-p)s = 1-q+qs = 1-q(1-s) 
 \end{align*}

Similarly. \( Y_n \) has generating function,
 \begin{align*} 
     G_Y(s) = \sum_{k=0}^{\infty} e^{-\lambda} \lambda^k/k! s^k = e^{-\lambda}e^{s\lambda} = e^{\lambda(s-1)}
 \end{align*} 
 

Therefore \( X_{n+1} \) has generating function,
\begin{align*}
    G_{n+1}(s) &= G_Y(s) \EE\left[ s^{M_{k,1}+M_{k,2}+...+M_{k,X_n}} \right] \\
    &= G_Y(s) \EE\left[\EE\left[ s^{M_{k,1}+M_{k,2}+...+M_{k,X_n}}\right] | X_n\right] \\
    &= G_Y(s) \EE \left[ (1-q(1-s))^{X_n} \right] \\
    &= G_Y(s) G_n(1-q(1-s)) \\
\end{align*}

First observe \( 1-q^i(1-(1-q(1-s))) = 1-q^{i+1}(1-s) \). We now use the relation \( G_{n+1}(s) = G_Y(s)G_n(1-q(1-s)) \) and the fact that \( G_0(s) = 1 \) to calculate,
\begin{align*}
    G_{n+1}(s) &= G_Y(s) G_n(1-q(1-s)) \\
%    &= G_Y(s) G_Y(1-q(1-s))G_{n-1}(1-q(1-(1-q(1-s)))) \\
    &= G_Y(s) G_Y(1-q(1-s))G_{n-1}(1-q^2(1-s)) \\
    &= G_Y(s) G_Y(1-q(1-s)) G_Y(1-q^2(1-s)) G_{n-2}(1-q^3(1-s)) \\
    &\mathrel{\makebox[\widthof{=}]{\vdots}} \\
    &= \prod_{i=0}^{n}G_Y(1-q^i(1-s)) 
\end{align*}

Then,
\begin{align*}
    \lim_{n\to\infty} G_{n}(s) &= \lim_{n\to\infty} G_{n+1}(s) \\
    &= \lim_{n\to\infty} \prod_{i=0}^{n}G_Y(1-q^i(1-s)) \\
    &= \lim_{n\to\infty} \prod_{i=0}^{n} \exp\left( \lambda(-q^i(1-s)) \right)\\
    &= \exp \left( \sum_{i=0}^{\infty} \lambda(-q^i(1-s) \right) \\
    &= \exp \left( \lambda(s-1) \dfrac{1}{1-q}\right) \\
    &= \exp \left( \dfrac{\lambda}{p}(s-1) \right) \\
\end{align*}

Thus, \( G_n(S) \) converges to the generating function of a Poisson random variable with parameter \( \lambda/p \).

Then \( X_n \) converges to a random variable distributed like a Poisson random variable with parameter \( \lambda/p \). The random variable for which \( X_n \) converges to must be the variable corresponding to the stationary distribution. Therefore, the stationary distribution is distributed like the probability mass function of this random variable. That is,
\begin{align*}
    \pi(k) = e^{-\lambda/p} \dfrac{(\lambda/p)^k}{k!}
\end{align*}

In the limit \( p\to 1 \), where we correct all mistakes, the stationary distribution looks like a Poisson distribution with parameter \( \lambda \). In the limit \( \lambda \to 0 \) so we do not make any new mistakes, \( \pi(0)\to 1 \) as expected.
\end{solution}


\begin{problem}[Exercise 4.7]
Give an example of a transition matrix \( P \) that admits multiple stationary distributions \( \pi \).
\end{problem}

\begin{solution}[Solution]
Define \( P \) to be the identity matrix. Then any distribution is a stationary distribution.
\end{solution}


\begin{problem}[Exercise 4.8]
    A Markov chain on \( S=\{0,1,2,...,n\} \) has transition probabilities \( p(0,0) = 1-\lambda_0 \), \( p(i,i+1) = \lambda_i \) and \( p(i+1,i) = \mu_{i+1} \) for \( i=0,1,...,n-1 \), and \( p(n,n)=1-\mu_n \). Show that the process is reversible in equilibrium.
\end{problem}

\begin{solution}[Solution]
We assume all entries not specified are zero. (I heard this is the intent, however I wonder why we are given \( \mu_j\) when \( \mu_j=1-\lambda_j \) for all \( j \)). We write the matrix \( P \) as,

Write \( \mu_n=1-\lambda_n  \). Thus, \( \mu_i=1-\lambda_i \) for \( i=1, ..., n \) as the sum of each row must be 1 (making the assumption that all entries not specified at zero). 
    {\tiny
    \begin{align*}
    P = \left[\begin{array}{rrrrrr}
        1-\lambda_0 & \lambda_0 & & &\\
        \mu_1 & & \lambda_1 & & \\
        & \mu_2 & & \lambda_2 & \\
        & & \mu_3 \\\\
        & & & & & \lambda_{n-1} \\
        & & & & \mu_n & 1-\mu_n
    \end{array}\right]
      = \left[\begin{array}{rrrrrr}
        1-\lambda_0 & \lambda_0 & & &\\
        1-\lambda_1 & & \lambda_1 & & \\
        & 1-\lambda_2 & & \lambda_2 & \\
        && 1-\lambda_3\\\\
        & & & & & \lambda_{n-1} \\
        & & & & 1-\lambda_n & \lambda_n
    \end{array}\right]
\end{align*}
    }

This chain is irreducible and finite so a unique invariant distribution \( \pi \) exists. Write \( \pi=[\pi_0,\pi_1, ..., \pi_n] \). Then \( \pi P = \pi \). That is,
\begin{align*}
    \pi P = \left[\begin{array}{r}
        \pi_0(1-\lambda_0)+\pi_1(1-\lambda_1) \\
        \pi_0 \lambda_0 + \pi_2(1-\lambda_2) \\ 
        \pi_1\lambda_1+\pi_3(1-\lambda_3) \\
        \vdots \\ 
        \vdots \\
        \pi_{n-1}\lambda_{n-1}+\pi_n\lambda_n
    \end{array}\right]^T
= \left[\begin{array}{r}
    \pi_0 \\ 
    \pi_1 \\
    \pi_2 \\
    \vdots \\
    \pi_j \\
    \vdots \\
    \pi_n
\end{array}\right]^T
\end{align*}

\begin{align*}
    \pi_1 &= \lambda_0\pi_0 / (1-\lambda_1) 
        & \lambda_0\pi_0 &= \pi_1(1-\lambda_1) \\
    \pi_2 &= (\pi_1-\pi_0\lambda_0)/(1-\lambda_2) = \pi_1\lambda_1/(1-\lambda_2) 
        & \lambda_1\pi_1 &= \pi_2(1-\lambda_2) \\
    \pi_3 &= (\pi_2-\pi_1\lambda_1)/(1-\lambda_3) = \pi_2\lambda_2/(1-\lambda_3) 
        & \lambda_2\pi_2 &= \pi_3(1-\lambda_3) \\
    \vdots \\
    \pi_{j+1} &= (\pi_j-\pi_{j-1}\lambda_{j-1})/(1-\lambda_{j+1}) = \pi_j\lambda_j/(1-\lambda_{j+1}) & \lambda_j\pi_j &= \pi_{j+1}(1-\lambda_{j+1})\\
    \vdots \\
    \pi_n &= (\pi_{n-1}\lambda_{n-1})/(1-\lambda_n) 
        & \pi_{n-1}\lambda_{n-1} &= \pi_n(1-\lambda_n)
\end{align*}

Observing the equations on the right hand side we have that for \( i=1,2,...,n-1 \),
\begin{align*}
    \pi_{i}p(i,i+p) = \pi_{i+1}p(i+1,i)
\end{align*}

We now show the detail balance condition. In particular, we must show, 
\begin{align*}
    \pi_ip(i,j) = \pi_jp(j,i)  &&\text{ for all } i,j 
\end{align*}
However, for \( j\notin\{i-1,i+1\} \) we have \( p(i,j)=0 \). Therefore, for this matrix the previous condition is equivalent to 
\begin{align*}
    \pi_i p(i,i+1) = \pi_{i+1}p(i+1,i) &&\text{ for } i=1,2,..., n-1
\end{align*}

We have shown that these equations hold for all \( i=1,2,...,n-1 \). 

This proves \( \pi \) is in detailed balance with \( P \), and so this process is reversible in equilibrium. \qed

\end{solution}

